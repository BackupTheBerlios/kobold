%%%%%%%%%%%%%%%%%%%%%%%%%%%%%%%%%%%%%%%%%%%%%%%%%%%%%%%%%%%%%%%%%%%%%%%%%%%%%%%
%% StuPro A, Produktlinien (Kobold)
%% Team Werkbold
%% Angebot
%% $Id: begriffslexikon.tex,v 1.1 2004/02/05 12:57:31 garbeam Exp $
%%%%%%%%%%%%%%%%%%%%%%%%%%%%%%%%%%%%%%%%%%%%%%%%%%%%%%%%%%%%%%%%%%%%%%%%%%%%%%%
\newcommand{\begriff}[2]
{\item \bfseries{#1} \textnormal{#2}}

\chapter{Begriffslexikon}
\begin{itemize}

\begriff{Arbeitskopie}{Kopie, mit der gearbeitet wird und tats�chlich ver�ndert wird}

\begriff{Architektur}{Struktur von Software, die aus Komponenten und
Konnektoren besteht}

\begriff{Assets}{Wiederverwendbare Komponentennn, z.B. Source-Pakete, Klassen,
usw.}

\begriff{Auschecken}{CVS-Funktion zum Anfordern einer Arbeitskopie aus dem Repository}

\begriff{Core-Assets}{Abstrakte Architekturkomponenten}

\begriff{Core-Asset-Entwickler}{Softwareentwickler, der Core-Assets entwickelt}

\begriff{Core-Asset-Repository}{Entwicklungs-Repository (Arbeitskopie) der
Core-Assets}

\begriff{CVS}{Concurrent Versions System, ein Versionsverwaltungssystem}

\begriff{Deprecated}{Veraltet, missbilligend}

\begriff{Einchecken}{CVS-Funktion zum R�ckschreiben der Arbeitskopie-�nderungen
ins CVS-Repository}

\begriff{Erstentwicklung}{Entwicklung eines neuen Produkts bzw. eines neuen Core-Assets}

\begriff{Evolution�res Vorgehensmodell}{Inkrementelles Vorgehensmodell in der
Softwareentwicklung, �hnlich dem Spiralmodell nach B.W. Boehm}

\begriff{Iteration}{Phase im evolution�ren Vorgehensmodell}

\begriff{Kernkomponenten}{Siehe Core-Assets}

\begriff{Komponenten}{Bestandteile der Produktlinien- bzw. Produktarchitektur}

\begriff{Metainformation}{Zusatz-Informationen �ber Informationen}

\begriff{Module}{Komponenten auf Repository-Ebene}

\begriff{Produkt-Ingenieur}{PE, ein Software-Ingenieur mit Leitungs- und Entscheidungskompetenz 
bzgl. der technischen Realisierung des Produkts. Der PE hat den Produktlinien-Ingenieur als Vorgesetzten.}

\begriff{Produktlinien-Ingenieur}{PLE, Software-Ingenieur mit projekt�bergreifender 
Leitungs- und Entscheidungskompetenz bzgl. der technischen Realisierung 
und Pflege der Produktlinienarchitektur.}

\begriff{Produktarchitektur}{Software-Architektur eines Produktes}

\begriff{Produkt-Entwickler}{P, Softwareentwickler dessen Vorgesetzter ein
Produkt-Ingenieur ist}

\begriff{PL-Repository}{Repository, das vom PLE verwaltet wird, PE checken
Arbeitskopien von Core-Asset Varianten aus}

\begriff{Produkt-Repository}{Repository, das vom PE verwaltet wird, Entwickler
checken Arbeitskopien von Produkt�Komponenten aus}

\begriff{Prototyp}{Entwicklungsversion eines Programmes}

\begriff{Repository}{Datei-Verwaltung zur Versionsverwaltung}

\begriff{Softwarearchitektur}{siehe Architektur}

\begriff{Update}{CVS-Funktion, die die Arbeitskopie mit dem CVS-Repository
synchronisiert}

\begriff{Varianten}{Modifikationen von Core-Assets}

\begriff{Version}{Eindeutige Bezeichnung zu einem fixen Zeitpunkt}

\begriff{Wasserfall-Vorgehensmodell}{Klassisches Vorgehensmodell in der
Softwareentwicklung, das in Phasen aufgeteilt ist, die aufeinander folgen}

\begriff{Workflow}{Arbeits- bzw. Gesch�ftsprozess}

\end{itemize}

%%% Local Variables: 
%%% TeX-master: "angebot"
%%% End: 
%%% vim:tw=79:
