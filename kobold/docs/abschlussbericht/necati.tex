\section{Necati Aydin}

\begin{figure}[h!]
  \centering
  \includegraphics[width=7cm]{necati}
  \caption{Necati Aydin}
\end{figure}

Da ich, wie auch einige andere aus unserem Team, das StuProB dem StuProA vorziehen konnten, hatte ich schon einige Erfahrung aus dem StuProB gesammelt.  Aber verglichen mit meinem ersten Studienprojekt war Kobold um einiges komplexer, was sich in laufe des Projekts herausstellte.  

\subsection{T�tigkeiten}

Im Vorprojekt bestand unser Team aus Patrick, Anselm, Oli, Tammo und mir. In dieser Phase des Projekts waren wir im wesentlichen alle an der Analyse der Anforderungen und der Ausarbeitung eines Angebots beteiligt. \par
Nachdem wir den Zuschlag f�r das Projekt bekamen und die restlichen Teammitglieder hinzukamen, versuchten wir die Ergebnisse der Analysen vom Vorprojekt beider Teams zusammenzuf�hren. Dabei konnte man beobachten, wie unterschiedlich die Anforderungen aufgefasst wurden waren. Nachdem wir dann, so gut es ging, auf einen gemeinsamen Nenner kamen, konnte es dann auch mit der Spezifikation beginnen, an der wir auch noch alle beteiligt waren. \par
In den fr�hen Phasen des Hauptprojekts nahm ich an den Reviews f�r den Projektplan und der Spezifikation teil. In der ersten Iteration war meine Aufgabe den "'Roletree"'(Navigationsbaum) zu implementieren. Dies erwies sich schwieriger, als ich es im Vorfeld angenommen hatte. Zum einen war ich mit Eclipse noch nicht so vertraut und zum anderen waren in dieser Phase noch Entwurfsfragen bez�glich des Datenmodells noch offen. \par
In der zweiten Iteration arbeitete ich am Dokument-Generator, der f�r ein beliebiges Asset alle Komponenten, Varianten, Benutzer, etc. ausliest und daraus ein pdf-Dokument generiert. Hierzu musste ich mich erst einmal mit "'iText"' besch�ftigen und mich hier einarbeiten. Dies war dann auch mein Schulungsthema. Gegen Ende der zweiten Iteration f�hrten wir dann ein Code-Review durch, an dem ich als Gutachter teilnahm. \par
Am Anfang der dritten und letzten Iteration (geplant waren zuerst vier), war ich immer �fter mit der Erstellung und Versch�nerung von Dialogen besch�ftigt. Zudem wurden mir immer wieder Bugs zugeteilt, die ich dann auch bearbeiten konnte. Zu dieser Phase des Projekts gab es auch keine klaren Zuordnungen mehr. Die Tasks die keinem Projektmitglied direkt zugeteilt waren, konnten von jedem bearbeitet werden, der gerade etwas Zeit hatte. \par
In einer sp�teren Phase des Projekts f�hrte ich Systemtests mit anderen Teammitgliedern zusammen durch. Die hier aufgefundenen Fehler wurden dann auch gleich in Bugtracker aufgef�hrt. Es war auch interessant, mal den Entwicklungsfortschritt und die Realisierung des Projekts des "'Konkurrenz"'-Teams zu sehen. Dies war mir geg�nnt, da ich mit einem weiteren Teammitglied, ihr Tool (PLAM) testen konnte. Hierbei konnte ich feststellen wie unterschiedlich Kobold und PLAM sind.


\subsection{Beurteilung}

Trotz einiger Schwierigkeiten war Kobold ein Erfolg. Durch dieses Projekt konnte ich sicherlich mehr Erfahrungen sammeln, die f�r mein Studium und f�r andere Projekte sehr hilfreich sein k�nnen. Zum Erfahrungsgewinn z�hlt f�r mich das Erlernen neuer Technologien, wie Eclipse, iText, etc., und das freundschaftliche Miteinander der Teammitglieder, obwohl man sich vorher nicht kannte. \par 
Die zwischenzeitlichen Motivationsprobleme, die durch die untersch�tzte Komplexit�t von Eclipse hervorgerufen wurde, wurden schnell behoben, indem die Teammitglieder immer wieder bereit waren sich gegenseitig zu helfen. Diese gegenseitige Hilfsbereitschaft verst�rkte den Zusammenhalt des Teams. Zudem war die Atmosph�re im Team immer sehr locker. \par

Vielen Dank an alle Beteiligten!

