\section{Abschlussbericht}

In diesem Projekt sammelte ich meine ersten Erfahrungen mit der eigenständigen Entwicklung von Software in einem größeren Team von Entwicklern. Ich arbeitete durch meine vorherige tätigkeit im Rahmen meines Werkstudenten Jobs schon mehrfach in sehr viel größeren Projekten mit, nie jedoch so eingebunden in den Entwicklungsprozess, von der Analyse bis zur Auslieferung. 
Die Seminarvorträge boten für mich eine wichtige Wissensgrundlage für die Entwicklung des Tools.


\subsection{Tätigkeiten}

Im Laufe des Projektes variierten meine Aufgaben ständig. So verhalf ich durch meine Vorherigen Erfahrung in der Plug-In Entwicklung für Eclipse, anhand von Hilfestellung und einem Vortrag den anderen Team Mitgliedern zu einem Überblick über die Grundsätzlichen Eclipse Eigenschaften und der Architektur des Systems. Danach wurde ich auch noch häufiger bei Fragen und Problemen im Bezug auf Eclipse herangezogen und versuchte zu helfen wo es möglich war. 
Nach dieser Anfangsphase erstellte ich zusammen mit Bettina Druckenm�ller die Spezifikation auf Basis der im Team und extern zusammengetragenen Anforderungen. Dies stellte sich als eine Herausforderung heraus, da nicht immer ganz klar war was sich genau aus den Anforderungen in die Spezifikation übernehmen lies. Auch waren einige Anforderungen bis kurz vor Schluss noch weitgehend ungeklärt. 
An die Spezifikation schloss sich dann relativ schnell die erste Implementierungsphase an in der ich zuerst verschiedene Funktionalitäten des PLAM Plug-Ins realisierte. Als nächstes fiel dann das VCM Plug-In meinen Aufgabenbereich. Ich erstellte einen Abstraktionslayer zur Eclipseinternen Versionskontrollinterface und implementierte ein Grobgerüst zu den vom VCM Plug-In zur Verfügung gestellten VCM Aktionen. Dazu gehörten auch die Integration der Actions in die Menüstruktur des Gesamtsystems. Im Laufe der Entwicklung wurden die meisten Aktionen jedoch aus dem Benutzermenü entfernt und automatisiert, so dass der Benutzer die einzelnen Menüeinträge nicht mehr sieht.
Um die Benutzerangaben die für das abarbeiten der VCM Aktionen nötig waren implementierte ich User Preferences und ein Preference Page zur Speicherung bzw. zur Änderung der Benutzerinformationen. 
Nach der Grobimplementierung schritt ich zur Ausarbeitung der
einzelnen Tasks fort und Oliver erstellte mit mir zusammen die für
die Funktionalität nötigen Schnittstellen f�r die extern
aufgerufenen Skripte.
Einige Schwierigkeiten bereiteten mir die Abarbeitung der verschiedenen externen Prozesse die für die Ausführung der Skripte benötigt wurden. Speziell die Kapselung in einzelnen Threads  und Prozesse wies sich als sehr komplex und Fehleranfällig auf. Durch ständige Kurztests der Funktionalität anhand einer Runtime-Workbench fielen mir schon nach den 1. Implementierung viele Fehler auf die ich sogleich korrigierte. Da ich aber unter Windows programmierte und Testete blieben mir viele Fehler verborgen die dann erst von Teamkollegen im Laufe der Zeit entdeckt wurden. Da sich das Datenmodell leider öfters änderte war ich dementsprechend oft und Zeitintensiv mit Änderungen und Anpassungen beschäftigt die nur den Zweck hatten Funktionalität wieder bzw. überhaupt erst herzustellen.

\subsection{Fazit}

Es traten wie in jedem größeren Projekt zu Erwarten einige Probleme auf, die zu Erwarten waren. So wurden meines Erachtens von uns allen die frühen Phasen unterschätzt und führten später bei der Implementation zu einigem Mehraufwand und Problemen. Was jedoch das gravierendste Problem dieses Projektes war, war der von Teammitglied zu Teammitglied sehr Unterschiedliche Arbeitseifer. Obwohl dies aus keinen der erhobenen Statistiken herauszulesen ist, war die Bereitschaft zur Einarbeitung und selbstständigen Problemlösung doch sehr unterschiedlich Ausgeprägt. Dies führet dazu, dass einige Teammitglieder sehr viel mehr leisteten als andere, und das man manchen Leuten regelrecht ihre Aufgaben „vorkauen“ musste. Relativiert wird dieses Argument durch den Fakt, dass alle Teammitglieder einen sehr Unterschiedlichen Wissensstand im Bezug auf die eingesetzten Technologien hatten, und Anfangs bei einigen eher Frust über die relativ komplexe Struktur des Projektes und seines Frameworks vorherrschte. 
Ich bin sehr Erfreut über die gute Atmosphere, die trotz der technischen Schwierigkeiten im Team herrschte und bin Überzeugt, dass jedes einzelne Teammitglied seine Erkenntnisse und Lehren aus diesem Projekt ziehen konnte. Leider konnten wir  nicht alles Umsetzen was wir uns zuerst Vorgestellt haben, aber die Grundfunktionaliät und einige nette Details sind letzlich doch verwirklichht worden. Meiner Meinung nach bietet das Tool sehr hohe Benutzerfreundlichkeit sowie eine zeitgemäßes User Interace und integriert sich auch Problemlos in bestehende Entwicklungsprozesse. Ein weiterer wichtiger Pluspunkt des Projektes ist die Plattformunabhängikeit die durch die Entwicklung in Java gewährleistet ist. Da die Sourcen für die Plattform und das Programm selber frei Verfügbar sind sollte eine unter Umständen gewünschte Anpassung keine größeren Probleme darstellen.
