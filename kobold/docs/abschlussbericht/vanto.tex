\section{Tammo van Lessen}
%foto
Wie f�r einige andere Teammitglieder ist das Studienprojekt A
``Kobold'' bereits das zweite Studienprojekt f�r mich gewesen.

\subsection{T�tigkeiten}
Nach der Projektvergabe bildeten Anselm, Patrick, Oli, Necati und ich
das ``Team Werkbold'' mit dem Ziel �ber das Vorprojekt den Zuschlag
f�r das Hauptprojekt zu bekommen.

\subsubsection{Vorprojekt}
Neben der aktiven Teilnahme an diversen Meetings zur Analyse der
Anforderungen interessierte mich besonders, wie man ein solches
Projekt am elegantesten angehen kann und welche Frameworks uns zur
Verf�gung stehen. Als langsam klar wurde, dass der Kunde nicht
ausschlie�lich auf eine Implementierung mit Ada mit GTK bestehen
w�rde, habe ich die Eclipseplattform als Alternative
vorgeschlagen. Da die Produktlinienverwaltung nicht nur f�r Manager
sondern gerade f�r Programmierer entwickelt werden sollte ist eine
integrierte Entwicklungsumgebung meines Erachtens das passende Framework f�r unser
Projekt. Mit dieser Begr�ndung, der Tatsache, dass einige im Team mehr
Erfahrung mit Java als mit Ada hatten und dass Eclipse auch
softwaretechnisch sehr interessant ist (Erich Gamma hat hier s�mtliche
seiner GoF-Patterns umgesetzt), stimmten die anderen
Teammitglieder schnell zu. Um sicher zu gehen, dass Eclipse auch
tats�chlich geeignet ist, sollte ein Prototyp entwickelt werden. So
war es nun meine Aufgabe anhand eines Prototypen die Machbarkeit des
Grapheneditors in der Eclipseplattform zu �berpr�fen. Dazu w�hlte ich GEF
(Graphical Editing Framework, http://www.eclipse.org/gef) als
Framework. Nach einiger Einarbeitung konnte ich dem Team den
Prototypen pr�sentieren. Da der Prototyp performant genug f�r die
Anforderungen war, war schnell unsere Entscheidung f�r Eclipse
gefallen. Nachdem die Anforderungen analysiert waren, formulierten wir
das Angebot. Daran waren Anselm und ich ma�geblich beteiligt.

Aus vier Angeboten entschied sich der Kunde f�r zwei Projekte. Dazu
geh�rte auch unser Projekt ``Kobold''. Unser Team wurde nun um vier
Teammitglieder erweitert.

\subsubsection{Hauptprojekt}
Die ersten Meetings nach der Zusammenlegung waren etwas schwieriger
als erwartet, denn die Vorstellungen beider Team von dem Projekt waren
doch etwas unterschiedlich. Nach l�ngerem hin und her einigten wir uns
und es wurde mit der Anforderungsdokumentation begonnen. Da die
Eclipseplattform sehr umfangreich und komplex ist, entschlossen wir
uns, uns gegenseitig zu schulen, damit wir alle einen m�glichst
gleichen Wissensstand zum Implementierungsbeginn haben. Ich bereitete
zwei Schulungen vor: In der ersten stellte ich im Rahmen eines
inf.misc-Vortrags das Build- und Projektmanagementtool ``Maven'' vor,
mit dessen Hilfe wir Metriken, Statistiken sowie die Projekthomepage
generieren. Der zweite Vortrag behandelte das Grapheneditorframework
GEF, dessen Implementierungsdetails und Patterns ich intern
vorstellte. Nach diesen Schulungen lag es auch nahe, dass ich w�hrend
des Entwurfs und der Implementierung f�r den graphischen Editor
zust�ndig war, den ich dann auch in Eigenregie entwickelte. Bevor es
dazu kam, stand erst der Entwurf des Datenmodells an auf dem die
anderen Komponenten dann operieren sollten. Im weiteren Verlauf stellte sich
jedoch heraus, dass das Modell nicht alle Anforderungen erf�llte, so
dass ich zusammen mit Martin die Architektur des Modells
�berarbeitete. 

Parallel zu der Entwicklung des Grapheneditors k�mmerte
ich mich um die Integration des Models in die Eclipse-Workbench,
programmierte den Navigationsbaum und die lokale Messageverwaltung.

\subsection{Beurteilung}
Zun�chst begann das Studienprojekt sehr viel
versprechend. Interessantes Thema, interessante Leute, interessante
Frameworks. Mit dem Entwurf zeigte sich allerdings, dass das
Eclipse-Framework vielleicht doch eine Nummer zu gro� f�r ein
Studienprojekt dieser Gr��enordnung war, da es eine intensive
Auseinandersetzung mit den dort eingesetzten Patterns und Mechanismen
erfordert, die in dem zeitlich begrenzten Rahmen leider nicht bei
allen m�glich war. Das war etwas ern�chternd. Diese Erfahrung ist aber
nur eine der vielen, die ich im
Verlauf des Projekts gemacht habe. Ich denke, dass gerade dies
die wichtigen Dinge sind, die man bei einem Studienprojekt mitnehmen
kann. So musste ich lernen, wie man mit unterschiedlichen
Arbeitsweisen und Wissensst�nden umgehen muss, wie man Konflikte im
Team vermeidet, wie wichtig Dokumentation ist, etc. Alles Dinge, die
man schon vorher wusste, aber anders priorisiert hat. Ich war
�berrascht, wie fatal die Auswirkungen von vermeintlichen
Kleinigkeiten sein k�nnen.

F�r mich war das Studienprojekt ein sehr erkenntnisreiches und spannendes
Studienprojekt, welches mir viel Spass aber auch einiges
Kopfzerbrechen �ber das Miteinander im Team bereitet hat. 

Trotzdem oder auch gerade deshalb sehe ich das Projekt als Erfolg an.
