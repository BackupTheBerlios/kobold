%%%%%%%%%%%%%%%%%%%%%%%%%%%%%%%%%%%%%%%%%%%%%%%%%%%%%%%%%%%%%%%%%%%%%%%%%%%%%%%
%% StuPro A, Produktlinien (Kobold)
%% Team Werkbold
%% Projektplan
%% $Id: organisation.tex,v 1.6 2004/02/02 11:00:54 neccaino Exp $
%%%%%%%%%%%%%%%%%%%%%%%%%%%%%%%%%%%%%%%%%%%%%%%%%%%%%%%%%%%%%%%%%%%%%%%%%%%%%%%

\chapter{Organisation}

\section{Teamorganisation}

Die Organisation des neuen Werkboldteams orientiert sich am
Vorbild des klassischen {\it Chief-Developer Teams}\footnote{Vgl.
Literatur, u.a. J. Ludewig, Software-Engineering f�r
Softwaretechniker, Vorlesungsfolien}, d.h. ein oder maximal zwei
Entwickler (Chief-Programmer) geben den roten Faden der
Entwicklung vor.

\section{Rollen}

Im Rahmen der Entwicklung von \product m�ssen von den
Werkbold-Mitarbeitern verschiedene Rollen wahrgenommen werden. Im
folgenden werden diese Rollen n�her beschrieben.

\subsection{Fixe Rollen}

Fixe Rollen werden einem konkreten Werkbold-Mitarbeiter f�r die
gesamte Dauer des \product-Projektes zugewiesen.

\begin{itemize}
\item {\bf Projektleiter:} Der Projektleiter vertritt die
Interessen des Teams nach au�en gegen�ber dem Auftraggeber,
koordiniert die Projektplanung, �berwacht den
Entwicklungsfortschritt und f�rdert den Informationsaustausch im
Team und zum Auftraggeber. Der Projektleiter f�r das
\product-Projekt ist Herr Anselm Garbe.\item {\bf Stellvertreter
des Projektleiters:} Sobald die Person des Projektleiters
verhindert ist (Terminkonflikte, Krankheit, Urlaub), �bernimmt der
Stellvertreter die zu erledigenden Aufgaben. F�r diesen Vorgang
ist insbesondere die Risikorichtlinie {\it Ausfall des
Projektleiters}\footnote{dieses Risikodokument ist noch zu
erstellen} zu beachten. Der stellvertretende Projektleiter f�r das
\product-Projekt wurde noch nicht benannt. \item {\bf
Configuration-Manager:} Der Configuration-Manager setzt die
Entwicklungsumgebung auf und stellt die Verwaltung und
Versionierung von Dokumenten und des Source-Codes sicher.\item
{\bf Dokumentations-Manager:} Der Dokumentations-Manager ist
technischer Berater f�r alle Dokumente, die in der Entwicklung
enstehen. Er ist f�r die Erstellung des Handbuchs verantwortlich.
\end{itemize}

\subsection{Dynamische Rollen}

Dynamische Rollen werden w�hrend der Entwicklung von den
entsprechenden Mitarbeitern �bernommen. In der Regel ist eine
solche Rollen�bernahme an eine bestimmte Aufgabe oder an ein
konkretes Arbeitspaket gebunden und zeitlich begrenzt.

\begin{itemize}
\item {\bf Chief-Developer:} Ein Chief-Developer ist w�hrend der
Entwicklung die f�r die Erf�llung einer bestimmten Aufgabe
verantwortliche Person. Er gibt den roten Faden f�r die Erf�llung
einer solchen Aufgabe vor (bspw. Erstellung des Entwurfs). Somit
fungiert er als technischer Berater f�r das Team und insbesondere
f�r die an der Erf�llung der spezifischen Aufgabe beteiligten
Entwickler. Bei �benahme dieser Rolle f�r eine Aufgabe wird vom
verantwortlichen Chief-Developer eine Kostensch�tzung und der
Arbeitsplan erstellt und w�hrend der Erf�llung gepflegt. Beachte
auch: {\it Richtline zur Durchf�hrung von
Chief-Development-Tasks}\footnote{dieses Richtliniendokument ist
noch zu erstellen}. \item {\bf Qualit�tssicherer:} Zum
Aufgabenbereich eines Qualit�tssicherers im \product-Projekt
geh�rt die Erstellung von Statements, Richtlinien-, Risiko- und
Schulungsdokumenten sowie die Organisation der praktischen
Durchf�hrung von Softwaretests. Dabei sind die entsprechenden
Richtlinien \footnote{Richtline zur Ertellung von Statements,
Richtline zur Erstellung von Richtliniendokumenten, Richtlinie zum
Risikomanagement, Richlinie zum Softwaretest - alle noch zu
erstellen} zu beachten. \item {\bf Entwickler:} Die Rolle des
Entwicklers wird von einem Werkbold-Mitarbeiter immer zusammen mit
einem vom Chief-Developer zugewiesenen Arbeitspaket �bernommen.
Entwicklungsarbeit in diesem Sinne ist also jedwede T�tigkeit zur
Erf�llung eines solchen Arbeitspaketes.

\end{itemize}

\section{Ansprechpartner}

\subsection{Kunde}
Auftraggeber:
\begin{itemize}
  \item Daniel Simon\\
    Telefon: +49-711-7816-213\\
    E-Mail: simon@informatik.uni-stuttgart.de\\
\end{itemize}

Qualit�tssicherung und Controlling:
\begin{itemize}
  \item Thomas Eisenbarth\\
    Telefon: +49-711-7816-345\\
    E-Mail: eisenbarth@informatik.uni-stuttgart.de\\
\end{itemize}

Technische Berater:
\begin{itemize}
  \item J�rg Czeranski\\
    Telefon: +49-711-7816-317\\
    E-Mail: czeranski@informatik.uni-stuttgart.de\\
  \item Gunther Vogel\\
    Telefon: +49-711-7816-375\\
    E-Mail: gunther.vogel@informatik.uni-stuttgart.de\\
\end{itemize}


\subsection{Werkbold-Team}

Projektmanagement:
\begin{itemize}

  \item Anselm Garbe\\
    Telefon: +49-711-8822280\\
    E-Mail: anselmg@t-online.de

\end{itemize}

Experten:
\begin{itemize}

  \item Armin Cont\\
    Telefon: +49-177-3915355\\
    E-Mail: neccaino@onlinehome.de

  \item Bettina Druckenm�ller\\
    Telefon: +49-163-4601901\\
    E-Mail: Bettina.Druckenmueller@t-online.de

  \item Martin Plies\\
    Telefon: +49-160-3709240\\
    E-Mail: martin.plies@web.de

  \item Michael Grosse\\
    Telefon: +49-7031-415940, +49-177-4974364\\
    E-Mail: mig.grosse@gmx.de

  \item Necati Aydin\\
    Telefon: +49-177-6029410\\
    E-Mail: neco.a@gmx.de

  \item Oliver Rendgen\\
    Telefon: +49-711-717619, +49-173-4363622\\
    E-Mail: stuproa@hypeo.de

  \item Patrick Schneider\\
    Telefon: +49-163-2788161\\
    E-Mail: PatrickM.Schneider@gmx.de

  \item Tammo van Lessen\\
    Telefon: +49-711-6586471, +49-171-2774106\\
    E-Mail: tvanlessen@taval.de

\end{itemize}

\subsection{Projekthomepage}

Die Projekthomepage von \product ist unter {\bf
http://kobold.berlios.de} abrufbar und enth�lt transparent zur
Entwicklung diverse Informationen zum Project Tracing, zu Terminen
und Entwickler-Howtos.
