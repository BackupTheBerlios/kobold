%%%%%%%%%%%%%%%%%%%%%%%%%%%%%%%%%%%%%%%%%%%%%%%%%%%%%%%%%%%%%%%%%%%%%%%%%%%%%%%
%% StuPro A, Produktlinien (Kobold)
%% Team Werkbold
%% Projektplan
%% $Id: entweinleit.tex,v 1.3 2004/02/02 07:47:49 garbeam Exp $
%%%%%%%%%%%%%%%%%%%%%%%%%%%%%%%%%%%%%%%%%%%%%%%%%%%%%%%%%%%%%%%%%%%%%%%%%%%%%%%

\chapter{Entwicklungsprozess}

In diesem Kapitel erfolgt die Planung der einzelnen
Entwicklungsiterationen des Produktlinien Management Systems
\product. F�r jede Iteration
werden die jeweilige Zielsetzung, deren Zeitplanung und
Meilensteine festgesetzt.
\section{Vorgehensmodell}

Es wird ein evolution�res Vorgehensmodell\footnote{Vgl. Literatur, u.a. J.
Ludewig, Sammlung von Kapiteln zum Software Engineering} angewendet, das sich
in mehrere Iterationen unterteilt. Jede Iteration entspricht dem
Wasserfall-Vorgehensmodell. Das Vorgehensmodell �hnelt dem Spiralmodell nach
B.W. Boehm.
Durch die Anwendung dieses Vorgehensmodells ergeben sich folgende Vorteile:

\begin{itemize}
\item Der Umfang der iterativen Spezifikationen und Entw�rfe ist wesentlich
kleiner als bei einer Wasserfall-Gesamtentwicklung. Der inkrementelle Umfang
resultiert in einer geringeren Fehleranf�lligkeit und garantiert somit einen
h�heren Qualit�tsstandard.
\item Der Entwicklungsfortschritt kann somit vom Auftraggeber und vom
Entwicklungsteam realistischer eingesch�tzt werden, da regelm��ig
eine neue Auslieferung durchgef�hrt wird, die vom Auftraggeber abgenommen
werden muss.
\item Durch die Wahl der {\it J2SE} Technologie findet die Entwicklung
garantiert nach modernen {\it Design Patterns}\footnote{Vgl. Literatur, Design
Patterns, Erich Gamma} und objektorientierten Paradigmen statt. Im Gegensatz zu
imperativen Technologien wird dadurch eine geringere Kopplung der
Software-Komponenten und somit eine bessere Zerlegung des Systems erreicht.
Eine derart leicht durchf�hrbare Zerlegung des Systems nach funktionalen
Bestandteilen erm�glicht besonders das evolution�re Vorgehen.
\item Es k�nnen neue oder ge�nderte Anforderungen in einer sp�teren Iteration
wesentlich flexibler im Entwicklunsprozess ber�cksichtigt und realisiert
werden.
\end{itemize}

Ein entscheidener Unterschied zu konventionellen Vorgehensmodellen ist der
h�here Kommunikationsaufwand zwischen dem Auftraggeber und dem Entwicklerteam.
Dieser h�here Kommunikationsaufwand zeichnet sich in jedem Fall positiv f�r den
Auftraggeber aus, da Anforderungen schon aufgrund der Quantit�t der
Kommunikation besser einfliessen werden.\par
Da f�r den Auftraggeber nach jeder
abgeschlossenen Iteration die M�glichkeit besteht, gr��ere
�nderungsw�nsche einzubringen, ist eine �ber die unmittelbar
n�chste Iteration hinausgehende Planung mit sehr gro�en
Unw�gbarkeiten verbunden. Die genaue Planung einer Iteration
erfolgt daher erst unmittelbar vor deren Beginn.


\section{Vorprojekt}

Im Vorprojekt wurde das vom Auftraggeber bereits angenommene Angebots-Dokument
erstellt und in der zweiten Kalenderwoche 2004 eine Angebotspr�sentation durchgef�hrt,
die den \product Prototypen dem Auftraggeber pr�sentierte.\par
Durch die Auftragserteilung in der dritten Kalenderwoche 2004 wurde das
CURES-Team in das Werkbold-Team integriert, das nun aus 9 Entwicklern
besteht. Mit der Auftragserteilung und der Neuformierung des
Werkbold-Teams endete das Vorprojekt. Durch die im Vorprojekt
durchgef�hrte Zeiterfassung beider Teams konnte ein genauer Aufwand
ermittelt werden, der im folgenden aufgef�hrt wird.

\subsection{Aufw�nde des Vorprojektes}
TODO
