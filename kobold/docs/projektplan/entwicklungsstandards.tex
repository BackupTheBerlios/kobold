%%%%%%%%%%%%%%%%%%%%%%%%%%%%%%%%%%%%%%%%%%%%%%%%%%%%%%%%%%%%%%%%%%%%%%%%%%%%%%%
%% StuPro A, Produktlinien (Kobold)
%% Team Werkbold
%% Projektplan
%% $Id: entwicklungsstandards.tex,v 1.3 2004/02/02 08:39:13 garbeam Exp $
%%%%%%%%%%%%%%%%%%%%%%%%%%%%%%%%%%%%%%%%%%%%%%%%%%%%%%%%%%%%%%%%%%%%%%%%%%%%%%%

\chapter{Entwicklungsstandards}

In diesem Kapitel werden die f�r die Entwicklung von \product
festgesetzten Standards dokumentiert. Durch die Einf�hrung
bindender Standards f�r Schulungen, Projekt Tracing,
Kundenintegration sowie Risiko- und Richtlinienmanagemnet wird
eine sehr solide Grundlage f�r \product Produkt- und
Projketqualit�tssicherung gelegt.

\section{Risikomanagement}

Verantwortungsbewusstes Risikomanagment ist eine permanente,
projektbegleitende T�tigkeit und kann somit nur schwer an vorab
festgelegten Zeitpunkten erfolgen. F�r die Entwicklung von
\product erfolgt diese T�tigkeit dennoch standardisiert in drei
Schritten:\par Nach der Identifikation der Risiken (1. Schritt)
erfolgt eine Einsch�tzung ihrer Wahrscheinlichkeit und ihres
Schadenspotentials (2. Schritt) als Grundlage f�r die Erarbeitung
von Strategien zur Schadensbegrenzung f�r den Fall ihres
Eintretens (3.Schritt).\par Ergebnis dieser T�tigkeit ist ein
Risikodokument, das die erarbeiteten Strategien in Form von
Notfall-Checklisten f�r den Ernstfall dokumentiert. Nach Ma�gabe
der Qualit�tssicherung werden diese Dokumente einem kurzen Review
durch das Team unterzogen. \par In jedem Fall sind die in den
Risikodokumenten beschriebenen Verhaltensweisen f�r alle
Teammitglieder bindend.

\section{Richtlinien}

�hnlich den Risikodokumenten werden auch im Rahmen des
Richtlinienmanagements (nach Ma�gabe der Qualit�tssicherung bzw.
Projketleitung) spezifische Richtliniendokumente erstellt. \par
Dabei wird die Ausgestaltung dieser Dokumente als direkt
anwendbare Checklisten angestrebt, bspw. f�r die Qualit�ttspr�fung
am Ende einer Iteration oder f�r das Design eines bestimmten
Dokuments.

\section{Kundenintegration}

Bei der Entwicklung von \product nimmt die Integration des Kunden
eine zentrale Rolle ein: Nach Abschluss jeder Iteration werden dem
Kunden die erzielten Ergebnisse vorgelegt. Bestandteil dieser
Ergebnisse ist immer ein ausf�hrbares System, auf dessen Grundlage
die weitere Evolution des Systems vom Kunden fr�hzeitig
beeinflusst werden kann. \par Dazu werden dem Kunden zusammen mit
den Ergebnissen auch detailierte Vorschl�ge f�r die in der
n�chsten Iteration umsetzbaren Anforderungen unterbreitet.
Dar�berhinaus hat der Kunde im Rahmen dieses Verfahrens die
M�glichkeit, mit der Projektleitung �ber die Aufnahme neuer
Anforderungen in die laufende Entwicklung zu verhandeln.

\section{Schulungen}

F�r die konkrete Ausgestaltung von Team-internen
Schulungsma�nahmen werden keine spezifischen Standards erhoben.
Allerdings ist binnen festzulegender Frist eine f�r alle
Teammitglieder zug�ngliche, schriftliche Ausarbeitung der im
Rahmen der Schulung vermittelten Informationen zu erstellen.

\section{Lizensierung}

In Absprache mit dem Auftraggeber in den Nachverhandlungen zum
abgenommenen Angebot wurde vereinbart \product unter den Bedingungen der
MIT License zu lizensieren. Nachfolgend ist die vom \product Team
verwendete Lizenz aufgef�hrt:\par

{\footnotesize
\begin{verbatim}
Copyright (c) 2004 Armin Cont, Anselm R. Garbe, Bettina Druckenmueller,
                   Martin Plies, Michael Grosse, Necati Aydin,
                   Oliver Rendgen, Patrick Schneider, Tammo van Lessen

Permission is hereby granted, free of charge, to any person obtaining a
copy of this software and associated documentation files (the "Software"),
to deal in the Software without restriction, including without limitation
the rights to use, copy, modify, merge, publish, distribute, sublicense,
and/or sell copies of the Software, and to permit persons to whom the
Software is furnished to do so, subject to the following conditions:

The above copyright notice and this permission notice shall be included
in all copies or substantial portions of the Software.

THE SOFTWARE IS PROVIDED "AS IS", WITHOUT WARRANTY OF ANY KIND, EXPRESS OR
IMPLIED, INCLUDING BUT NOT LIMITED TO THE WARRANTIES OF MERCHANTABILITY,
FITNESS FOR A PARTICULAR PURPOSE AND NONINFRINGEMENT. IN NO EVENT SHALL
THE AUTHORS OR COPYRIGHT HOLDERS BE LIABLE FOR ANY CLAIM, DAMAGES OR OTHER
LIABILITY, WHETHER IN AN ACTION OF CONTRACT, TORT OR OTHERWISE,
ARISING FROM, OUT OF OR IN CONNECTION WITH THE SOFTWARE OR THE USE OR
OTHER DEALINGS IN THE SOFTWARE.
\end{verbatim}
}
