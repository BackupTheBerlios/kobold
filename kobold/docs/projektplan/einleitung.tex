%%%%%%%%%%%%%%%%%%%%%%%%%%%%%%%%%%%%%%%%%%%%%%%%%%%%%%%%%%%%%%%%%%%%%%%%%%%%%%%
%% StuPro A, Produktlinien (Kobold)
%% Team Werkbold
%% Angebot
%% $Id: einleitung.tex,v 1.2 2004/01/29 11:23:32 garbeam Exp $
%%%%%%%%%%%%%%%%%%%%%%%%%%%%%%%%%%%%%%%%%%%%%%%%%%%%%%%%%%%%%%%%%%%%%%%%%%%%%%%

\chapter{Einleitung}

Dieses Dokument beschreibt den Prozess, die Organisation
und die Standards zur Entwicklung des Produktlinien Management
Systems \product, um den angebotenen Leistungskatalog zu realisieren.
Es wird dem Auftraggeber als Projektplan f�r das Hauptprojekt vorgelegt.
\par
Das wesentliche Einsatzziel des Produktlininen Management Systems \product
ist die werkzeugunterst�tzte Entwicklung und Pflege von
Softwareproduktlinien und die Etablierung eines rollenbasierten
Entwicklungsprozesses. Dieser soll eine effiziente
Produktlinienentwicklung erm�glichen und wird durch \product unterst�tzt.

\section{G�ltigkeit}

Aufgrund des evolution�ren Vorgehensmodells wird die iterative 
Prozessbeschreibung am Anfang jeder Folge-Iteration verfeinert. Dies
resultiert in einer iterativen Vervollst�ndigung und Auslieferung
des Projektplans am Anfang aller Folge-Iterationen. D.h. eine
detaillierte Arbeitsplanung f�r die jeweilige Folge-Iteration ist
jeweils aus der verfeinerten Version dieses Projektplans ersichtlich,
die zum jeweiligen Folge-Iterationsbeginn dem Auftraggeber ausgeliefert
wird.\par
Dieser Projektplan ist in der groben Iterationsplanung jedoch f�r das
gesamte Hauptprojekt g�ltig. Durch die verfeinerte Arbeitsplanung der
Iteration I ist dieser Projektplan f�r diese Iteration bindend.

\section{Entwicklungsphilosophie}

Die Entwicklungsphilosophie folgt dem evolution�ren Vorgehensmodell,
das sich in mehrere Iterationen unterteilt. Jede Iteration folgt dem
Wasserfall-Vorgehensmodell.\par
Die erste Iteration, die in verfeinertem Ma�e Gegenstand der
Arbeitsplanung in diesem Projektplan ist, gew�hrleistet die Erstellung
eines soliden Rahmensystems und stellt die Basisfunktionalit�t gem��
den vereinbarten Anforderungen zur Verf�gung.\par
In den weiteren Iterationen werden die kundenspezifischen Anforderungen nach
einer unter Entwicklungsstandards bestimmten Priorisierung realisiert.

\section{Hintergrund der Produktlinienentwicklung}

Schl"usselerfolgsfaktoren in der Softwaretechnik sind kurze
Time-to-market Zyklen, hohe Produktqualit"at und niedrige Kosten. Das Erzielen
dieser scheinbar unvereinbaren Ziele wird durch systematische
Wiederverwendung w"ahrend der Entwicklung von Software m"oglich.\par
Kernpunkt des Produktlinienansatzes ist die systematische Erfassung
von Unterschieden und Gemeinsamkeiten der verschiedenen Produkte und
einer expliziten Abbildung dieser Aspekte in die
Softwarearchitektur. Diese Grundidee hat Auswirkungen f"ur alle
Schritte im Produktentwicklungsprozess und - idealerweise - auch
dar"uber hinaus in der Produktplanung und -strategie.\par
Um eine effiziente Produktlinienentwicklung in der Softwareentwicklung
zu etablieren, ist es wichtig, dass diese Produktstrategie konsequent
umgesetzt wird.\par
Idealziel einer gut konfigurierten Produktlinie ist es, ein Produkt
durch einfache Kombination von angepassten Core-Assets zu
erstellen. Eine Produktlinie ist also ideal konfiguriert, wenn die
Produktlinien-Architektur f�r alle m�glichen Produkte hinreichend
spezifiziert ist. Der Produkt\-linien-Ingenieur hat genau
diese Aufgabe. Er bestimmt die Architektur in dem er Core-Assets
definiert und deren Grundbeziehungen zueinander spezifiziert (Scoping
und Domain-Engineering). Soll nun ein neues Produkt in die
Produktlinie aufgenommen werden (Application Engineering), so muss es
dieser Architektur folgen. Es werden dazu bereits existierende
Varianten der Core-Assets mit neuen Varianten kombiniert und dem
Produkt-Ingenieur �bergeben, der die Verantwortung f�r das Produkt
�bernimmmt. Dieser hat daf�r Sorge zu tragen, dass die Entwicklung an
dem Produkt nicht die Architektur der Produktlinie verletzt.\par 
In dieser Hierarchie ist es wichtig, dass bestimmte Vorg�nge nach
festen Regeln kommuniziert werden. Wenn z.B. in einem Produkt eine
bereits existierende Variante eines Core-Assets verwendet wird, darf
diese nicht von dem Produkt-Entwickler ver�ndert werden. Ver�nderungen
sind hier nur dem zust�ndigen Core-Asset-Entwickler gestattet. Dieser
muss abw�gen, ob die Ver�nderung sinnvoll ist, da er die
Verantwortung f�r sein Modul, welches evtl. auch noch von anderen
Produkten verwendet wird, hat. Lehnt er dies ab, so muss der
Produkt-Ingenieur eine produktspezifische Variante dieses Core-Assets
entwickeln lassen. Dieser Vorgang ist sehr komplex, da mehrere Personen
mit ihren Entscheidungen daran beteiligt sind. Um eine Produktlinie
konsequent durchsetzen zu k�nnen, m�ssen diese Arbeitsabl�ufe
spezifiziert sein.\par
Es bietet sich nun an, die Entwicklung von Produktlinien mit Werkzeugen zu
unterst�tzen. Damit kann man sowohl das Configuration Managment
vereinfachen, sowie die oben erw�hnten Arbeitsabl�ufe modellieren und damit
die Kommunikation und die Entwicklung erleichtern.\par
Gefordert ist also ein Werkzeug, das es m�glich macht eine Architektur f�r
eine Produktlinie zu entwerfen, die f�r alle an der Entwicklung beteiligten
Stellen bindend ist.\par

\section{Gegenstand}
Mit dem Auftraggeber wurde der Leistungskatalog aus unserem Angebot zur
Entwicklung des Produktlinien Managment Systems \product vereinbart, dass die
im vorherigen Abschnitt beschriebenen Anforderungen umsetzen wird.\par
Gegenstand dieses Projektplans ist die prozesstechnische Realisierung des
\product Produktlinien Managment Systems.

%%% vim:tw=79:
