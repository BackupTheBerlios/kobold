%%%%%%%%%%%%%%%%%%%%%%%%%%%%%%%%%%%%%%%%%%%%%%%%%%%%%%%%%%%%%%%%%%%%%%%%%%%%%%%
%% StuPro A, Produktlinien (Kobold)
%% Team Werkbold
%% Angebot
%% $Id: iterationweitere.tex,v 1.8 2004/07/25 22:40:05 garbeam Exp $
%%%%%%%%%%%%%%%%%%%%%%%%%%%%%%%%%%%%%%%%%%%%%%%%%%%%%%%%%%%%%%%%%%%%%%%%%%%%%%%

\section{Nachbesserungsphase}

F�r die Nachbesserungsphase werden abz�glich der Aufw�nde der
durchgef�hrten Iterationen I-III {\bf 742} Arbeitsstunden als
Puffer verwendet (das entspricht den 5 Wochen vom 09.08.2004 - 15.09.2004).
Sollten
vom Auftraggeber keine Nachbesserungsw�nsche gefordert werden, k�nnen
neue Anforderungen f�r den Aufwand von {\bf 742} Arbeitsstunden
beauftragt werden.

\subsection{Projektende}

Am Ende der Nachbesserungsphase wird eine endg�ltige Produktauslieferung
des \product Systems durchgef�hrt, die alle nachgebesserten bzw. w�hrend
der Nachbesserungsphase implementierten Zusatzanforderungen umfasst.

Das Projektende wird durch eine Abschlusspr�sentation feierlich markiert.

\subsection{Lieferumfang}

Am Projektende wird das endg�ltige \product System sowie alle
zugeh�rigen und w�hrend der Entwicklung entstandenen Dokumente
ausgeliefert, sowie ein Projekt-Abschlussbericht, der die
Projektdurchf�hrung in der Nachbetrachtung noch einmal Revue passieren
l��t und eine individuelle Projektkritik aller Werkbold-Mitarbeiter
enth�lt.
