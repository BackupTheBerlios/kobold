%%%%%%%%%%%%%%%%%%%%%%%%%%%%%%%%%%%%%%%%%%%%%%%%%%%%%%%%%%%%%%%%%%%%%%%%%%%%%%%
%% StuPro A, Produktlinien (Kobold)
%% Team Werkbold
%% Angebot
%% $Id: iterationweitere.tex,v 1.1 2004/02/02 10:17:35 garbeam Exp $
%%%%%%%%%%%%%%%%%%%%%%%%%%%%%%%%%%%%%%%%%%%%%%%%%%%%%%%%%%%%%%%%%%%%%%%%%%%%%%%

\section{Folge-Iterationen}

In den Folge-Iterationen werden in Kommunikation mit dem Auftraggeber nach
einer Priorisierung alle noch offenen oder neue Anforderungen
realisiert. U.a. folgende:

\begin{itemize}
\item Administrationstool f�r den Server (kommandozeilen-orientiert)
\item Statusdarstellung des Servers (Web-basiert)
\item Workflow Definition und Realisierung (Client-Server Kommunikation)
\item Architektur-Verwaltungskomponente (Repository gekoppelt)
\item usw.
\end{itemize}

Aufgrund des in jeder Iteration durchzuf�hrenden Wasserfall-Vorgehensmodells
werden auch in jeder Folge-Iteration folgende zugeh�rige Dokumente erstellt und
ausgeliefert:

\begin{itemize}
\item Spezifikation der zu implementierenden Komponenten
\item Entwurf der zu implementierenden Komponenten
\item Source- und Bytecode der implementierten Komponenten
\item Source- und Bytecode aller notwendigen Komponenten von Drittherstellern
\item Entwickler-Howtos zu den implementierten Komponenten (sofern vorhanden)
\item Testf�lle, Testplan und Testresultate zu den implementierten Komponenten
\end{itemize}

\subsection {Zeitplan}

Angestrebt werden im Rahmen dieses Projekplanes drei
Folge-Iterationen, die sich wie folgt in den Zeitplan integrieren:

\begin{tabular}{llrll}\\
\bf ID & \bf Arbeitspaket & \bf Aufwand & \bf Termin & \bf Meilenstein \\ \hline
24 & Erstellung Spezifikation & 150 & & \\
25 & Review Spezifikation & 50 &  & \\
26 & Korrektur Spezifikation & 20 &  & \\
27 & Auslieferung Spezifikation & & & M\*\\
28 & Erstellung Entwurf / Refactoring & 150 & & \\
29 & Implementierung / Dokumentation & 250 & & \\
30 & Test und Korrektur & 50 & & \\
31 & Auslieferung des erweiterten Systems & 30 & & M\* \\ \hline
   & \bf Gesamt & \bf 700 & 4-5 Wochen &\\
\end{tabular}

Die genauen Termine der Meilensteine in den Folge-Iterationen werden dem
Auftraggeber im angepassten Projektplan vor Beginn der jeweiligen
Folge-Iteration mitgeteilt.\par
Werden pro Folge-Iteration vier bis f�nf Wochen Arbeitsaufwand veranschlagt,
so ergeben sich im Mittel f�r die einzelnen Team-Mitglieder in dieser Zeit
etwa 15 Arbeitsstunden pro Woche. Dies erm�glicht die rechtzeitige
Auslieferung des Endproduktes von \product am {\bf 13.08.2004}.\par
In der folgenden Tabelle sind die Iterationen zeitlich nach
Kalenderwochen eingegrenzt:

\begin{tabular}{lll}
\bf Iteration & \bf Beginn & \bf Ende \\ \hline
Vorprojekt & 06.11.2003 & KW2\\
Iteration I & KW4 (2004) & KW20 (14.05.2004)\\
Iteration II & KW21 (2004 & KW25 (18.06.2004)\\
Iteration III & KW26 (2004) & KW30 (23.07.2004)\\
Iteration IV & KW30 (2004) & KW33 (13.08.2004)\\
\end{tabular}

\subsection{Kostensch�tzung}
TODO

\subsection{Meilensteine}

Das Projekt ist in mehrere Iterationen
unterteilt, die sich wiederum in kleinere Phasen aufteilen. Phasen
werden im Rahmen dieses Projektplans durch (externe) Meilensteine
abgeschlossen. Meilensteine gelten erst als erreicht, wenn die f�r die
jeweilige Phase geplanten Arbeitspakete zur vollst�ndigen Zufriedenheit
des Auftraggebers erfolgreich erstellt und ausgeliefert wurden.\par
Erst nach dem erreichen eines Meilensteins wird die Folgephase
begonnen.\par
Folgende Meilensteine werden in der Projektdurchf�hrung
verwirklicht:\par
\begin{tabular}{lll}
\bf Iteration & \bf Meilensteine \\ \hline
Vorprojekt & M1 (Angebot), M2 (Angebotspr�sentation)\\
Iteration I & M3 (Projektplan), M4 (Spezifikation), M5 (Entwurf),\\
            & M6 (Rahmensystem)\\
Iteration II & M7 (Spezifikation), M8 (Verfeinertes System)\\
Iteration III & M9 (Spezifikation), M10 (Verfeinertes System)\\
Iteration IV & M11 (Spezifikation), M12 (Endprodukt)\\
\end{tabular}

Die Meilensteine der Folge-Iterationen sind bis dato nur sehr unscharf geplant,
sie werden im Rahmen der Revision des Projektplans vor Beginn der
jeweiligen Folge-Iteration konkretisiert.\par
Ein ungef�hrer �berblick �ber den Inhalt der Meilensteine in den
Folge-Iterationen vermittelt folgender Abschnitt:

\subsubsection{M7, M9, M11: Spezifikationen}

{\bf Ergebnis:} Spezifikation von priorisierten Anforderungen\par
In der jeweiligen Iterationen wird eine Spezifikation erstellt, die als
Grundlage f�r die evolution�re Verfeinerung und Vervollst�ndigung des
bis dahin erstellten Systems dient. Jede Spezifikation enth�lt folgende
anforderungs-spezifischen Abschnitte:
\begin{itemize}
\item Funktionale Anforderungen
\item Use-Cases
\item Nicht-Funktionale Anforderungen (soweit notwenig)
\item Begriffslexikon
\end{itemize}
Zur Endauslieferung wird eine Gesamt-Spezifikation ausgeliefert, die
alle Teil-Spezifikationen enth�lt.\par
Alle Spezifikationen in den jeweiligen Iterationen werden im Rahmen
eines Reviews in Zusammenarbeit mit dem Auftraggeber gepr�ft.

\subsection{M8, M10, M12: Verfeinertes System}

{\bf Ergebnis: } Lauff�higes und getestetes System\par
Diese Meilensteine markieren das jeweilige Ende der zugeh�rigen
Iteration. Ziel ist es, ein lauff�higes und zuverl�ssiges System zu
liefern, das alle f�r die jeweilige Iteration priorisierten
Anforderungen enth�lt. Dar�ber hinaus werden neben den zugeh�rigen Source-
und Bytecodes (inkl. notwendiger Komponenten von Drittanbietern) alle
relevanten Dokumente, Howtos, Testf�lle, Testpl�ne und Testresultate
ausgeliefert (Vgl. Lieferumfang), um die Qualit�t des jeweils ausgebauten
Systems zu unterstreichen.\par
In der letzten Iteration wird das vollst�ndige System inkl. aller unter
Lieferumfang genannten Dokumente und Software-Komponenten ausgeliefert.

\section{Nachbesserungsphase}

F�r die Nachbesserungsphase werden abz�glich der Aufw�nde der
dargestellten Folge-Iterationen {\bf 400} Arbeitsstunden als
Puffer verwendet (das entspricht den 4 Wochen vom 13.08.2004 - 10.09.2004).
Sollten
vom Auftraggeber keine Nachbesserungsw�nsche gefordert werden, k�nnen
neue Anforderungen f�r den Aufwand von {\bf 400} Arbeitsstunden
beauftragt werden.
