%%%%%%%%%%%%%%%%%%%%%%%%%%%%%%%%%%%%%%%%%%%%%%%%%%%%%%%%%%%%%%%%%%%%%%%%%%%%%%%
%% StuPro A, Produktlinien (Kobold)
%% Team Werkbold
%% Projektplan
%% $Id: iteration2.tex,v 1.1 2004/06/02 19:14:51 garbeam Exp $
%%%%%%%%%%%%%%%%%%%%%%%%%%%%%%%%%%%%%%%%%%%%%%%%%%%%%%%%%%%%%%%%%%%%%%%%%%%%%%%

\section{Iteration II}

Ziel der zweiten Iteration ist die Entwicklung und Auslieferung eines
lauff�higen Kobold Systems, dass alle im Folgenden genannten UseCases
implementiert. Darauf aufbauend werden in den Folge-Iterationen dar�ber
hinausgehende Anforderungen entwickelt.\par
Folgende UseCases sind f�r die Iteration II geplant:

\begin{itemize}
\item UC Anlegen/Modifizieren neuer Benutzer
\item UC Anlegen von Produkt
\item UC Modifizieren von Produkt/Produktlinie
\item UC VCM Aktionen von P/PL
\item UC Ausl�sen von Workflows an IKoboldServer
\item UC Administrationstool Kobold Server
\item UC GXL-Export inkl. JAR-Export
\end{itemize}

Dar�ber hinaus werden in der ersten Iterationen folgende Dokumente erstellt:

\begin{itemize}
\item Spezifikation der zu entwickelnden UseCases
\item Entwurf der UseCase-Implementierung
\item Entwickler Howtos zu UseCase-Implementierung
\item JUnit Testsuite
\item Teil-Handbuch zur den implementierten UseCases
\end{itemize}

\subsection {Zeitplan}

Durch den krankheitsbedingten Teilausfall zweier Team-Mitglieder konnte in der
Iteration I nicht der vorgesehene Gesamtaufwand von 1600h erreicht werden.
Demzufolge wurde der Aufwand der Iteration II etwas nach oben korrigiert, da
die ausgefallenen Team-Mitglieder einen h�herem Nachholaufwand in das Projekt 
investieren werden.

\begin{tabular}{llrll}\\
\bf ID & \bf Arbeitspaket & \bf Aufwand & \bf Termin & \bf Meilenstein \\ \hline
25 & Spezifikation II der zu entwickelnden UseCases & 100 & KW22 & \\
26 & Review der Spezifikation II & 50 & KW24 & \\
26 & Korrektur der Spezifikation II & 50 & KW24 & \\
27 & Auslieferung der Spezifikation II & & KW24 & M7 \\
26 & Implementierung der UseCases & 500 & KW26 & \\
26 & Test der UseCases & 50 & KW26 & \\
26 & Erweitertes Handbuch UseCases & 100 & KW26 & \\
26 & Auslieferung des zweiten Kobold Systems & & 24.6. (KW26) & M8 \\
   & \bf Gesamt & \bf 850 & 5 Wochen &\\
\end{tabular}

Sofern nicht n�her angegeben, sind Auslieferungstermine, die nach Kalenderwochen
angegeben wurden bis sp�testens zum letzten gesetzlichen Arbeitstag um {\bf 16:00
Uhr} in der jeweiligen Kalenderwoche einzuhalten. Sofern eine Auslieferung zu
einem fr�heren Zeitpunkt m�glich ist, wird diese an einem fr�heren Zeitpunkt in
R�cksprache mit dem Auftraggeber durchgef�hrt.\par
Sollte sich eine Auslieferung absehbar verz�gern, wird der Auftraggeber
m�glichst mind. eine Kalenderwoche im Voraus dar�ber in Kenntnis gesetzt.
Sollte sich eine Auslieferung nur kurzfristig (aufgrund technischer Probleme
oder durch Peronalausfall) verz�gern, d.h. auf den Folge-Arbeitstag um {\bf
8:00 Uhr} in der Folge-Kalenderwoche (also �bers Wochenende), so wird dies dem
Auftraggeber bis sp�testens eine Stunde vor dem sp�testm�glich vereinbarten
Auslieferungstermin bekanntgegeben.

\subsection{Kostensch�tzung}

Die Sch�tzung im Voraus wird nach dem Top-Down Ansatz durchgef�hrt und
beruht vorrangig aus Bottom-Up Vergleichen aus der Ermittlung der bisher
angefallenen Aufw�nde zur Entwicklung des Produktlinien Management Systems
\product sowie auf Erfahrungswerten aus Industrie-Praktikas,
dem Softwarepraktikum, dem Studienprojekt B (Automatisierung) und der
Initiierung verschiedener
umfangreicher OpenSource Projekte im C++ und J2SE-Umfeld.\par
Nach Abzug der Aufw�nde f�r das Vorprojekt und der Iteration I 
und unter Zugrundelegung eines Gesamtaufwands f�r die Projektdurchf�hrung
von {\bf 500 Stunden} pro Team-Mitglied (entspricht {\bf 4500 Stunden}
Gesamtaufwand) verbleiben f�r das Hauptprojekt {\bf XXXX Stunden}.\par
Der Iteration II wird nach Anpassung der Detailaufw�ndeein Aufwand von
{\bf 850 Stunden} zugrundegelegt, dies entspricht einem prozentualem Anteil
von ca. {\bf 25\%} vom Gesamtaufwand, wenn {\bf 4500 Stunden} f�r den
Gesamtaufwand angesetzt werden. Legt man einer Arbeitsstunde
einen Preis von {\bf 100 EUR} zugrunde, entspricht dies {\bf 85.000 EUR}.
Die Kosten f�r die einzelnen Arbeitspakete ergeben sich aus dem Produkt des
Aufwands (siehe Zeitplan) multipliziert mit dem Preis einer Arbeitsstunde.

\subsection{Meilensteine}
Das Projekt ist in mehrere Iterationen
unterteilt, die sich wiederum in kleinere Phasen aufteilen. Phasen
werden im Rahmen dieses Projektplans durch (externe) Meilensteine
abgeschlossen. Meilensteine gelten erst als erreicht, wenn die f�r die
jeweilige Phase geplanten Arbeitspakete zur vollst�ndigen Zufriedenheit
des Auftraggebers erfolgreich erstellt und ausgeliefert wurden.\par
Erst nach dem erreichen eines Meilensteins wird die Folgephase
begonnen.\par

In der zweiten Iteration werden folgende Meilensteine verwirklicht:

\subsubsection{M7: Spezifikation}

{\bf Ergebnis:} Spezifikation von UseCases \par
Die Spezifikation von priorisierten UseCases dient als Grundlage zur Entwicklung
aller erweiterten Funktionen des Basissystems f�r \product.\par
Im Einzelnen enth�lt die Spezifikation folgende Abschnitte:
\begin{itemize}
\item Funktionale Anforderungen an UseCases
\item Begriffslexikon bzgl. der UseCases
\end{itemize}
Die Spezifikation wird im Rahmen eines Reviews in Zusammenarbeit mit dem
Auftraggeber gepr�ft.

\subsubsection{M8: Erweitertetes Kobold System}

{\bf Ergebnis: } Lauff�higes und um UseCases erweitertes Rahmensystem\par
Die Auslieferung des erweiterten Rahmensystems stellt das Ende der
Iteration II dar.
Ziel ist es ein lauff�higes und zuverl�ssiges System zu erstellen,
dass die spezifizierten UseCases (priorisierte Anforderungen) aufsetzend
auf das bereits vorhandene Rahmensystem implementiert. 

Die Auslieferung des erweiterten Rahmensystems enth�lt (wie unter Lieferumfang
angegeben), neben dem Source- und Bytecode (inkl. notwendiger
Komponenten von Drittanbietern) auch eine
Installationsanleitung sowie alle Testf�lle, den Testplan und die
Testresultate. Anhand der Testresultate wird die Qualit�t des
erweiterten Rahmensystems unterstrichen.

\subsection{Lieferumfang}

Ziel der Iteration II ist die Auslieferung eines lauff�higen erweiterten Rahmensystems,
das die priorisierten UseCases verwirklicht. Im Einzelnen werden folgende Dokumente
und Software-Pakete ausgeliefert:

\begin{itemize}
\item Spezifikation der UserCases
\item Entwurf der UseCases
\item Source- und Bytecode des Clients und Servers
\item Source- und Bytecode notwendiger Komponenten von Drittanbietern
\item Testf�lle, Testplan und Testresultate des Clients und Servers
\item Teil-Handbuch der UseCases
\end{itemize}

Alle anderen Dokumente, die w�hrend dieser Iterationen entstehen, geh�ren nicht zum Lieferumfang.


%%% Local Variables: 
%%% TeX-master: "angebot"
%%% End: 
%%% vim:tw=79:
