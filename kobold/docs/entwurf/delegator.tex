\section{Delegator-Komponente Serverkommunikation}

Die clientseitige Delegator-Komponente fungiert als Implementierung der
Serverschnittstelle und dient zur Delegation aller serverseitigen
Methodenaufrufe. Neben der Delegierung der Methodenaufrufe an den Server
via XML-RPC kapselt die Delegator-Komponente auch das Berechtigungskonzept
zum Server von den restlichen clientseitigen Komponenten ab. Die
Delegator-Komponente ist ein Singleton.

\subsection{Kapselung des Berechtigungskonzeptes}

Der Delegator meldet sich mit Benutzernamen und Passwort am Server an und
erh�lt vom Server eine spezifische Session-ID. Danach werden vom
Server alle Konfigurationsdaten geladen und automatisch mit der
vorhandenen Konfiguration der Eclipse Entwicklungsumgebung abgeglichen.

\subsubsection{Automatischer Login bei wiederholtem Zugriff}

Beim ersten Login am Server wird von diesem eine eindeutige Session-ID
vergeben, die bei erneutem Zugriff auf den Server zur
Authentifizierung genutzt wird. 

Falls die selben Benutzerdaten von verschiedenen Clients zur
Authentifizierung am Server verwendet werden, wird die erste bereits
vergebene Session-ID ung�ltig und eine neue Session-ID an die zweite
Client Instanz vergeben. Dadurch wird die Datenkonsistenz
sichergestellt.

\subsection{Resultatwerte der XML-RPC Aufrufe auf dem Server}

Durch die Auswertung der Resultatwerte von XML-RPCs kann die
Delegator-Komponente entscheiden, ob die Server-Aktion erfolgreich
ausgef�hrt wurde oder ob eine erneute Authentifizierung notwendig ist.
