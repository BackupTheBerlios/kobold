
\section{Use Cases der Benutzeraktionen}
Use Cases - Anlegen / Modifizieren User


Use Case Name      : Neuen Benutzer anlegen\\
Akteur             : PE des Produktes\\
Vorbedingung       : Benutzer ist mindestens PE; Benutzername noch nicht vorhanden; ein Produkt ist ausgew�hlt\\
Regul�rer Ablauf   : \\
Der Benutzer klickt mit der rechten Maustaste auf das Produkt, in welchem der Benutzer hinzugef�gt werden soll.
 Es erscheint das Kontextmen�, hier klickt er auf Neu->Benutzer.
Es erscheint nun ein Wizard in dem der Benutzer aufgefordert wird, den Namen und ein Initialpasswort f�r den neuen 
Benutzer festzulegen.Mit Klick auf 'Next' wird der anlegende Benutzer mittels einer ComboBox aufgefordert, die Rolle 
innerhalb des aktuellen Projektes festzulegen. Ein weiterer Klick auf 'Next' schliesst den Wizard und 
legt den Benutzer an.\\
Nachbedingung      : \\
\begin{itemize}
\item Benutzer ist angelegt
\item Benutzer hat im gew�hlten Produkt die gew�nschte Rolle
\item Benutzer erscheint in allen Auflistungen der Benutzer
\end{itemize}
Alternativer Ablauf: keiner \par



Use Case Name		: Rolle �ndern\\
Akteur				: PE des Produktes\\
Vorbedingung		:\begin{itemize}
\item Anlegender ist mindestens PE
\item Benutzer ist vorhanden
\item ein Produkt ist ausgew�hlt
\end{itemize}

Regul�rer Ablauf	:\\
Der Benutzer klickt mit der rechten Maustaste auf das Produkt, in welchem der Person eine Rolle zugewiesen werden soll.
 Es erscheint das Kontextmen� in welchem er 'Rolle �ndern' anklickt.
Es erscheint ein Wizard, mit einem Texteingabefeld und 3 Listboxen, zudem 4 Buttons. In der linken Liste befinden sich 
alle verf�gbaren Benutzernamen. In den beiden Listboxen auf der rechten Seite werden alle PE / Ps des Produktes in der 
Liste angezeigt. Die 4 Buttons symbolisieren das "R�berschieben in die jeweils andere Liste" ("<<" bzw. ">>"). Das 
Eingabefeld dient der linken Gesamtbenutzerliste als Filterkriterium. Werden hier Buchstaben eingegeben erscheinen nur 
noch die Benutzernamen in der darunterliegenden Liste, die mit diesen Buchstaben anfangen.
Wenn ein Benutzer auf die jeweils andere Seite geschoben werden soll, muss sein Benutzernamen in der jeweiligen Liste
 markiert und dann der zugeh?�rige Button gedr?�ckt werden. Danach klickt der Benutzer wieder auf OK um das Fenster zu 
verlassen.\\

Nachbedingung		:\\
Benutzer hat neue Rolle \par





Use Case Name		: Benutzer l�schen\\
Wollen wir das wirklich per Client erm�glichen?\par



Use Case Name		: Passwort �ndern\\
Akteur				: Benutzer
Vorbedingung		: Benutzer ist vorhanden\\

Ablauf:\\
Der Benutzer klickt auf ein noch n�her zu spezifizierendes Ding. Es erscheint ein Dialog in welchem er sein altes
 Passwort noch einmal eingeben muss und danach sein neues Passwort doppelt eingeben muss. Der Dialog endet mit OK.
 Bei Klick auf Abbrechen wird der Vorgang abgebrochen und das Passwort nicht ge�ndert.\\

Nachbedingung		:\\
Passwort f�r Benutzer ist ge�ndert.\par