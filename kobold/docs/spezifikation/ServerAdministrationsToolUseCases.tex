\section{Use Cases des Server-Administrations-Tools}

Use Case Name      : Fehleingabe t�tigen\\
Akteur             : Kobold Administrator\\
Vorbedingung       : Der Kobold Server ist gestartet, die Eingabeaufforderung der Administartionskonsole ist zu sehen.\\
Regul�rer Ablauf   : \\
\begin{itemize}
\item Der Kobold Administartor gibt ein ung�ltiges Kommando ein.
\item Eine Fehlermeldung wird auf der Konsole ausgegeben.
\end{itemize}
Nachbedingung      : Die Eingabeaufforderung der
Administartionskonsole ist zu sehen.\par

Use Case Name      : Kommando�bersicht anzeigen\\
Akteur             : Kobold Administartor\\
Vorbedingung       : Der Kobold Server ist gestartet, die Eingabeaufforderung der Administartionskonsole ist zu sehen.\\
Regul�rer Ablauf   : \\
\begin{itemize}
\item Der Koboldadministartor gibt das Kommando "help" ein.
\item Eine �bersicht �ber die verf�gbaren Kommandos erscheint auf
der Konsole, mit einer kurzen Beschreibung jedes Kommandos.
\end{itemize}
Nachbedingung      : Die Eingabeaufforderung der
Administartionskonsole ist zu sehen.\par

Use Case Name      : Neue Produktlinie anlegen\\
Akteur             : Kobold Administrator\\
Vorbedingung       : Der Kobold Server ist gestartet, die Eingabeaufforderung der Administartionskonsole ist zu sehen.\\
Regul�rer Ablauf   : \\
\begin{itemize}
\item Der Koboldadministartor gibt das Kommando "create
productline" ein. \item Der Koboldadministartor wird aufgefordert
einen Namen f�r die neue Produktlinie anzugeben. \item Der
Koboldadministrator gibt einen g�ltigen Namen f�r die neue
Produktlinie an. \item Eine Erfolgsmeldung wird ausgegeben.
\end{itemize}
Alternative Abl�ufe: \\
Falls der eingegebene Name ung�ltig ist (leere Eingabe,
Produktlinie bereits vorhanden), wird eine entsprechende
Fehlermeldung ausgegeben und die Bearbeitung abgebrochen.\\
Nachbedingung      : Die Eingabeaufforderung der
Administartionskonsole ist zu sehen.\par

Use Case Name      : Produktlinie entfernen\\
Akteur             : Kobold Administrator\\
Vorbedingung       : Der Kobold Server ist gestartet, die Eingabeaufforderung der Administartionskonsole ist zu sehen.\\
Regul�rer Ablauf   : \\
\begin{itemize}
\item Der Koboldadministartor gibt das Kommando "remove
productline" ein. \item Der Koboldadministrator wird aufgefordert
den Namen der zu entfernenden Produktlinie einzugeben. \item Der
Kobold Administrator gibt einen g�ltigen Produktliniennamen an.
\item Die angegebene Produktlinie wird aus dem System eintfernt
und eine kurze Erfolgsmeldung ausgegeben.
\end{itemize}
Alternative Abl�ufe: \\
Falls der eingegebene Produktlinien-Name ung�ltig ist (leere
Eingabe, Produktlinie nicht vorhanden), wird eine entsprechende
Fehlermeldung ausgegeben und die Bearbeitung abgebrochen.\\
Nachbedingung      : Die Eingabeaufforderung der
Administartionskonsole ist zu sehen.\par

Use Case Name      : Neuen PLE anlegen\\
Akteur             : Kobold Administrator\\
Vorbedingung       : Der Kobold Server ist gestartet, die Eingabeaufforderung der Administartionskonsole ist zu sehen.\\
Regul�rer Ablauf   : \\
\begin{itemize}
\item Der Koboldadministartor gibt das Kommando "add ple" ein.
\item Der Koboldadministrator wird aufgefordert den Namen des neu
anzulegenden PLE einzugeben. \item Der Kobold Administrator gibt
einen g�ltigen PLE-Namen an. \item Der Kobold Administartor wird
aufgefordert ein Passwort f�r den neuen PLE anzugeben.\item Der
Kobold Administrator gibt ein Passwort ein.\item Ein neuer PLE mit
dem angegebenen Namen wird im System angelegt.
\end{itemize}
Alternative Abl�ufe: \\
Falls der eingegebene PLE-Name ung�ltig ist (leere Eingabe, Name
bereits vergeben), wird eine entsprechende
Fehlermeldung ausgegeben und die Bearbeitung abgebrochen.\\
Nachbedingung      : Die Eingabeaufforderung der
Administartionskonsole ist zu sehen.\par

Use Case Name      : PLE aus System entfernen\\
Akteur             : Kobold Administrator\\
Vorbedingung       : Der Kobold Server ist gestartet, die Eingabeaufforderung der Administartionskonsole ist zu sehen.\\
Regul�rer Ablauf   : \\
\begin{itemize}
\item Der Koboldadministartor gibt das Kommando "remove ple" ein.
\item Der Koboldadministrator wird aufgefordert den Namen des zu
entfernenden PLEs einzugeben. \item Der Kobold Administrator gibt
einen g�ltigen PLE-Namen an. \item Der angegebene PLE wird aus dem
System entfernt.
\end{itemize}
Alternative Abl�ufe: \\
Falls der eingegebene PLE-Name ung�ltig ist (leere Eingabe, Name
nicht vergeben), wird eine entsprechende
Fehlermeldung ausgegeben und die Bearbeitung abgebrochen.\\
Nachbedingung      : Die Eingabeaufforderung der
Administartionskonsole ist zu sehen.\par

Use Case Name      : PLE einer Produktlinie zuweisen\\
Akteur             : Kobold Administrator\\
Vorbedingung       : Der Kobold Server ist gestartet, die Eingabeaufforderung der Administartionskonsole ist zu sehen.\\
Regul�rer Ablauf   : \\
\begin{itemize}
\item Der Koboldadministartor gibt das Kommando "assign ple" ein.
\item Der Koboldadministrator wird aufgefordert den Namen des
zuzuordnenden PLEs einzugeben. \item Der Kobold Administrator gibt
einen g�ltigen PLE-Namen an. \item Der Kobold Administartor wird
aufgefordert den Namen der Produktlinie anzugeben, der der
angegebene PLE zugeordnet werden soll. \item Der Kobold
Administartor gibt einen g�ltigen PLE-Namen ein. \item Der PLE
wird der Produktlinie zugewiesen.
\end{itemize}
Alternative Abl�ufe: \\
Falls der eingegebene PLE-oder Produktlinienname ung�ltig ist
(leere Eingabe, Name nicht vergeben) , wird eine entsprechende
Fehlermeldung ausgegeben und die Bearbeitung abgebrochen.\\
Nachbedingung      : Die Eingabeaufforderung der
Administartionskonsole ist zu sehen.\par

Use Case Name      : Kobold Server beenden\\
Akteur             : Kobold Administrator\\
Vorbedingung       : Der Kobold Server ist gestartet, die Eingabeaufforderung der Administartionskonsole ist zu sehen.\\
Regul�rer Ablauf   : \\
\begin{itemize}
\item Der Koboldadministartor gibt das Kommando "exit" ein. \item
Der Kobold-Server wird beendet.
\end{itemize}
Nachbedingung      : Die Eingabeaufforderung der
Administartionskonsole ist nicht mehr zu sehen, Eingaben werden
von System nicht mehr entgegengenommen.\par
