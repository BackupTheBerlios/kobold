\section{Workflow manuell ausl�sen}

Use Case Name      : Nachricht schreiben\\
Akteur             : Benutzer\\
Vorbedingung       : Der Kobold Client ist gestartet, die Workflow/Task View ist zu sehen.\\
Regul�rer Ablauf   : \\
\begin{itemize}
\item Im Men� der Workflow/Task View wird der Punkt "Nachricht schreiben" ausgew�hlt.
\item Ein Dialog �ffnet sich.
\item Der Akteur gibt die zu versendende Nachricht und den Empf�nger an.
\item �ber den "Senden"-Button wird die Mail versendet; der "Abbrechen"-Button beendet die Aktion.
\item Es erfolgt eine Best�tigung bei erfolgreichem Versand.
\end{itemize}
Nachbedingung      : Der Dialog schlie�t sich; der Anfangszustand ist erreicht.\\
Alternativer Ablauf: Fehlt der Empf�nger der Nachricht, so erscheint eine Fehlermeldung.\par



Use Case Name      : Datei f�r eine Core Group vorschlagen\\
Akteur             : Benutzer\\
Erster Schritt     : \\
Vorbedingung       : Der Kobold Client ist gestartet, die Workflow/Task View ist zu sehen.\\
Regul�rer Ablauf   : \\
\begin{itemize}
\item Im Men� der Workflow/Task View wird der Punkt "Datei f�r Core Group vorschlagen" ausgew�hlt.
\item Ein Dialog �ffnet sich.
\item Der Akteur gibt den Namen und Pfad der Datei an und einen Kommentar.
\item �ber den "Vorschlagen"-Button wird die Workflow-Nachricht versendet; der "Abbrechen"-Button beendet die Aktion.
\item Es erfolgt eine Best�tigung bei erfolgreichem Versand.
\end{itemize}
Nachbedingung      : Der Dialog schlie�t sich; der Anfangszustand ist erreicht. 
Die Workflow-Nachricht wurde an den zust�ndigen PE geschickt.\\
Alternativer Ablauf: Fehlen die Angaben �ber die Datei, so erscheint eine Fehlermeldung.\par

Zweiter Schritt    : \\
Vorbedingung       : Der Kobold Client ist gestartet, die Workflow/Task View ist zu sehen; in ihr 
befindet sich die Nachricht aus Schritt 1, der Akteur ist der zust�ndige PE.\\
Regul�rer Ablauf   : \\
\begin{itemize}
\item Die Workflow-Nachricht wird doppelt-geklickt.
\item Ein Dialog �ffnet sich, in dem die Nachricht des Programmierers steht.
\item Der Akteur entscheidet �ber einen Radiobutton, ob er dem Vorschlag zustimmt oder ihn ablehnt.
\item In einem Textfeld kann er einen zus�tzlichen Kommentar eintragen.
\item �ber den "Senden"-Button wird die Workflow-Nachricht versendet; der "Abbrechen"-Button beendet die Aktion.
\item Es erfolgt eine Best�tigung bei erfolgreichem Versand.
\end{itemize}
Nachbedingung      : Der Dialog schlie�t sich; der Anfangszustand ist erreicht.
Bei Vorschlagsablehnung wird eine Nachricht an den Programmierer geschickt. Ansonsten wird die 
Workflow-Nachricht an den zust�ndigen PLE geschickt.\\
Alternativer Ablauf: Wurde kein Radio-Button selektiert, so erscheint eine Fehlermeldung.\par

Dritter Schritt    : \\
Vorbedingung       : Der Kobold Client ist gestartet, die Workflow/Task View ist zu sehen; in ihr 
befindet sich die Nachricht aus Schritt 2, der Akteur ist der PLE.\\
Regul�rer Ablauf   : \\
\begin{itemize}
\item Die Workflow-Nachricht wird doppelt-geklickt.
\item Ein Dialog �ffnet sich, in dem die Nachrichten des P und PE stehen.
\item Der Akteur entscheidet �ber einen Radiobutton, ob er dem Vorschlag zustimmt oder ihn ablehnt.
\item �ber den "Senden"-Button wird die Workflow-Nachricht versendet; der "Abbrechen"-Button beendet die Aktion.
\item Es erfolgt eine Best�tigung bei erfolgreichem Versand.
\end{itemize}
Nachbedingung      : Der Dialog schlie�t sich; der Anfangszustand ist erreicht.
Die Entscheidung des PLE wird in einer Nachricht an den PE und P geschickt.\\
Alternativer Ablauf: Wurde kein Radio-Button selektiert, so erscheint eine Fehlermeldung.\par













\section{Workflow automatisch ausl�sen}

Use Case Name      : Workflow automatisch ausl�sen\\
Akteur             : Kobold Client\\
Vorbedingung       : Der Kobold Client ist gestartet.\\
Regul�rer Ablauf   : \\
Die nachfolgenden Aktionen kreieren eine neue WorkflowMessage, die an den Server geschickt wird:
\begin{itemize}
\item commit
\item checkout
\item update
\item add to version control
\item delete (aus Repository)
\end{itemize}
Nachbedingung      : Eine WorkflowMessage wurde an den Server und somit an die WorkflowEngine 
weitergeleitet. Der Benutzer kann entsprechende Regeln f�r die Aktionen schreiben, die in
diesem Fall ausgef�hrt werden.\\
Alternativer Ablauf: -\par
