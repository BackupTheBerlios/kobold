\section{GXL Export}

Use Case Name      : GXL Export\\
Akteur             : Benutzer\\
Vorbedingung       : Der Kobold Client ist gestartet, eine Produktlinie oder eine Produkt ist als Graph zu sehen\\
Regul"arer Ablauf   : \\
\begin {itemize}
\item Der Benutzer klickt mit der rechten Maustaste auf in der Graphenview auf die Produktlinie, das Produkt, Komponente oder Variante.
\item Der Benutzer w"ahlt im Men"u ''GXL-Export'' aus.
\item Der Wizard GXL Export "offenet sich.
\item Der Benutzer w"ahlt aus, ob auch eine JAR-Datei mit dem Source erstellt werden soll.
\item Der Benutzer gibt den Pfad und Namen der GXL-Datei und des JAR-Archives an oder gibt die Daten im ''Speichern unter''-Dialog an.
\item Der Benutzer klickt auf ''Fertig''/''Finish''-Button.
\item Der Stand des Export wird im Wizard angezeigt.
\item Nach Fertigstellung des Exports, wird der Wizard beendet.
\end{itemize}
Nachbedingung      : Der Wizard ist geschlossen. Der Anfangszustand des Kobold clients ist wieder im Anfangszustand.\\
Alternativer Ablauf:\\
\begin{itemize}
\item Klickt der Benutzer auf ''Abbruch''/''Cancel''-Button, wird der Wizard vorzeitig geschlossen.\\
\item Tritt w"ahrend des Exports ein Fehler auf wird ein Dialog mit einer Fehlermeldung angezeigt und der Export abgebrochen.\\
\item Existieren die GXL/Jar-Datei mit dem angegegeben Namen schon wird der Benutzer ein Dilog gefragt, ob die Datei(en) "uberschrieben werden soll. Klickt der Benutzer auf den ''Ok''-Button, wird der Export erstellt.
\item Klickt der Benutzer auf den ''Abbruch''/''Cancel''-Button wird der GXL Export abgebrochen und der Wizard geschlossen.\\
\end{itemize} \par

