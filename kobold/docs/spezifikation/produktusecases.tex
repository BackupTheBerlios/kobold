\section{Architektureditor}

\subsection{Produkt}

\subsubsection{Neues Produkt anlegen}

\usecase{Kobold ist gestartet}
{Der Benutzer �ffnet die Palette im Architektureditor und w�hlt das Werkzeug 'compose product' aus. 
Daraufhin �ndern alle Objekte im Editor ihre Farbe in grau. Der Benutzer klick auf die Objekte, die er
in seinem Produkt enthalten haben m�chte. Diese f�rben sich daraufhin blau. Ist der Benutzer mit seiner
Auswahl zufrieden, bet�tigt er den 'create Product' Button im Editor. Ein Dialog �ffnet sich, in dem der 
Benutzer die Metainformationen f�r das neue Produkt eingeben kann.}
{Alle gew�hlten Module inklusive ihrer ben�tigten Referenzen sind ausgecheckt und werden dem
Beutzer grafisch angezeigt.}
{Nicht spezifiziert}
{PLE}


\subsubsection{Produkt umbenennen}
\usecase{Es ist ein Produkt ausgew�hlt}
{Der Benutzer doppelklickt auf das Produkt im RoleTree. Ein Dialog �ffnet sich, in dem er den Namen
 �ndern kann. Der Benutzer best�tigt mit 'OK'.}
{Produkt ist umbenannt.}
{Nicht spezifiziert}
{PE}


%\subsubsection{Produkt �ndern}
%\usecase{
%Es ist ein Produkt selektiert und der
%Benutzer ist mindestens PE des Produktes
%}
%{
%Der Benutzer bekommt den Wizard wie in Produkt erstellen zu sehen, allerdings erst ab 
%Schritt 2. Alle von ihm ausgew�hlten Einstellungen sind �bernommen.}
%{
%Benutzer hat die von ihm gew�nschten �nderungen vorgenommen,
%angezeigte Daten sind aktualisiert}
%{Nicht spezifiziert}
%{PE}

\subsubsection{Produkt als veraltet markieren}
\usecase{
Benutzer ist mindestens PLE f�r diese Produktlinie}
{
Der Benutzer doppeltklickt auf das entsprechende Produkt. Es �ffnet sich ein Dialog,
in dem er das Produkt auf deprecated setzen kann. Klickt er hier auf OK ist
das Produkt auf deprecated gesetzt und seine Module stehen den PEs nicht
mehr zur Auswahl.}
{
Produkt ist auf deprecated gesetzt, seine Module stehen f�r neue Produkte nicht mehr zur 
Verf�gung.}
{Nicht spezifiziert}
{PLE}
