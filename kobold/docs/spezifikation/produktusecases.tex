\section{Use Cases der Produktaktionen}


Use Case Name		: Neues Produkt anlegen\\
Akteur				: PLE\\
Vorbedingung		: Kobold ist gestartet\\

Ablauf:\\
Der Benutzer klickt irgendwo auf die Action Neues Produkt erstellen. Ein Wizard �ffnet sich
und zeigt eine Liste der f�r den Benutzer m�glichen Produktlinien aus. Der Benutzer w�hlt
nun eine der ihm zu Verf�gung stehenden Produktlinien aus und klickt auf 'Weiter'. 
Im n�chsten Schritt bekommt er eine Liste aller CoreAssets zur Verf�gung, welche Standard-
m�ssig alle per H�ckchen selektiert sind. Sollte er eines hiervon nicht w�nschen, so
klickt er einfach auf das entsprechende H�ckchen. Sobald er damit fertig ist, klickt er 
wiederrum auf 'Weiter'.
Im darauffolgenden Schritt kann er sich aus den f�r diese Produktlinie bestehenden Produkten
noch weitere Module aussuchen. 
Die Auswahl sieht folgenderma�en aus. Alle Produkte dieser Produktlinie sind als Dateibaum
dargestellt und der Benutzer w�hlt die von ihm gew�nschten Teilmodule aus. Neben dem Dateibaum
wird dem Benutzer eine kurze Beschreibung zum jeweils gew�hlten Modul angezeigt.
Es ist m�glich mehrere Module verschiedener Produkte gleichzeitig auszuw�hlen. Au�erdem sind
alle von den gew�hlten Modulen ben�tigten Produkte automatisch mit ausgew�hlt.
Klickt der Benutzer nun auf 'Fertigstellen', werden die entsprechenden Module geladen, 
ausgecheckt und dem Benutzer grafisch angezeigt.\\

Nachbedingung:\\
Alle gew�hlten Module inklusive ihrer ben�tigten Referenzen sind ausgecheckt und werden dem
Beutzer grafisch angezeigt.\par



Use Case Name		: Produkt umbenennen\\
Akteur				: PE des Produktes\\
Vorbedingung		: Es ist ein Produkt ausgew�hlt\\

Ablauf:\\
Der Benutzer klickt auf die rechte Maustaste und das Kontextmen� �ffnet sich. Hier wird
der Punkt Produkt umbenennen angezeigt. W�hlt er diesen aus, so kann er das Produkt umbenennen,
indem er einen neuen Namen eingibt. Der Name wird nun sowohl im Repository, als auch im
Produktlinienbaum aktualisiert.\\

Nachbedingung:\\
Produkt ist umbenannt.\par



Use Case Name		: Produkt �ndern\\
Akteur				: PE des Produktes\\
Vorraussetzung		:
\begin{itemize}
\item Es ist ein Produkt gew�hlt
\item Benutzer ist mindestens PE des Produktes
\end {itemize}\par

Ablauf:\\
Der Benutzer bekommt den Wizard wie in Produkt erstellen zu sehen, allerdings erst ab 
Schritt 2. Alle von ihm ausgew�hlten Einstellungen sind �bernommen.\\

Nachbedingung:\\
\begin {itemize}
\item Benutzer hat die von ihm gew�nschten �nderungen vorgenommen
\item Angezeigte Daten sind aktualisiert
end {itemize}\par



Use Case Name		: Produkt als veraltet markieren\\
Akteur				: PLE einer Produktlinie\\
Vorraussetzung		:
Benutzer ist mindestens PLE f�r diese Produktlinie\\

Ablauf:\\
Der Benutzer klickt mit der rechten Maustaste auf das entsprechende Produkt. Innerhalb des
Kontextmen�s hat er den Punkt 'Produkt als veraltet markieren' zur Auswahl. Bei Klick auf
den Button erscheint eine Messagebox mit nochmaliger Nachfrage. Klickt er hier auf OK ist
das Produkt auf deprecated gesetzt und seine Module stehen den PEs nicht mehr zur Auswahl.\\

Nachbedingung:\\
Produkt ist auf deprecated gesetzt, seine Module stehen f�r neue Produkte nicht mehr zur 
Verf�gung.\par



Use Case Name		: Modul auf veraltet setzen\\
Akteur				: PE eines Produkts\\
Vorraussetzung		:
enutzer ist mindestens PE des Produktes\\

Ablauf:\\
Der Benutzer klickt mit der rechten Maustaste auf das Modul. Innerhalb des Kontextmen�s
hat er den Punkt 'Modul auf deprecated setzen' zur Auswahl. Bei Klick auf diesen Punkt
erscheint eine Messagebox und bei neuerlichem best�tigen ist das Modul auf deprecated 
gesetzt und steht f�r neue Produkte nicht mehr zur Auswahl.\\

Nachbedingung:\\
Modul ist auf deprecated gesetzt und steht f�r neue Produkte nicht mehr zur Auswahl.\par
