\section{Kobold Serverdienst}

Der Kobold Serverdienst koordiniert die Arbeit der Benutzer von \product.
Ein Benutzer ist dabei jede Person mit einem Benutzernamen und dem dazu
passenden Passwort. Dieser ist dadurch bei seiner Arbeit nicht an einen
festen Rechner gebunden, sondern kann von jedem Rechner mit einem Kobold
Client �ber den Kobold Serverdienst an seine Daten gelangen und diese
bearbeiten. Ausserdem erm�glicht der Server eine
Client-zu-Client Kommunikation �ber eine Nachrichten-Queue, welche
im Minutentakt von den Clients abgefragt wird.\par

Der Server wird als HTTPS-basierter {\it XML-RPC Server}
\footnote{N�here Informationen: http://ws.apache.org/xmlrpc/}
implementiert und bietet damit verschl�sselte Kommunikation.
Alle Informationen, die der Server bereitstellt, werden gespeichert.

\subsection{Berechtigungskonzept}

Zur Realisierung des produktlinien\-�bergreifenden Berechtigungskonzepts
verwaltet der Server alle benutzer-, rollen-, produktlinien- und
produktbasierten Berechtigungen. Dabei kann ein Server produktlinien-�bergreifend
verwendet werden.\par
Der Server bietet ein auf diesen Berechtigungen basierenden
Authentifizierungsmechanismus an, der von allen Clients verwendet wird. Wird ein Client gestartet, so muss
sich der Benutzer �ber einen Dialog beim Server mit Benutzername und Passwort anmelden. Sind beide korrekt,
so werden in seinen Client die Daten der ihm zugeteilten Rollen und den
dazu geh�renden Produktlinien bzw. Produkte geladen. Der Benutzer
kann dabei nur die Daten ver�ndern, f�r die er auch eine Berechtigung besitzt.

\subsection{Persistenz}

Die Speicherung aller f�r das Berechtigungskonzept notwendigen Daten wird
durch eine abstrakte, XML-basierte Persistenzschicht, realisiert, wodurch eine
flexible Datenhaltung erm�glicht wird.\par
Die Persistenzschicht dient zur Speicherung aller Rollen-,
Authentifizierungs-, Produktlinien- und Produktrelationen. Die Grundstruktur dieser Schicht ist
baumartig aufgebaut. Die im Speicher aktuelle Schicht wird nach Ablauf eines konfigurierbaren Zeitintervalls
gespeichert (standardm��ig alle 10 Minuten).\par
In der XML-Struktur werden hierbei die folgenden Elemente gespeichert:
\begin{itemize}
\item Benutzer
\item Rollen
\item Repositories
\item Produktlinien
\item Produkte
\item Relationen
\end{itemize}

F�r jeden Benutzer werden vom Kobold Serverdienst folgende Daten gespeichert:
\begin{itemize}
\item Benutzername
\item Passwort
\item eMail
\item Name
\end{itemize}


\subsection{Nachrichten-Queue}

Der Kobold Serverdienst bildet f�r jeden Kobold-Benutzer eine
Nachrichten-Queue. Werden Nachrichten oder Workflow-Objekte an
einen Benutzer geschickt, so werden diese in der Nachrichten-Queue
gespeichert und bei der n�chsten Nachrichtenabfrage des Benutzers an ihn
weitergeleitet.\par Die vom Server angebotene Nachrichten-Queue
ist zustandsbehaftet, das hei�t dass bei nicht physisch begr�ndeten
Ausf�llen die Nachrichten-Queue in der Regel ohne Datenverlust
wiederherstellbar ist.\par

F�r jede Nachricht in der Nachrichten-Queue werden folgende Daten ben�tigt:
\begin{itemize}
\item Absender
\item Empf�nger
\item Inhalt
\item Datum
\item ID
\item Priorit�t
\end{itemize}

\subsubsection{Workflow-Mechanismus auf dem Kobold Server}
Sendet ein Client dem Server eine Nachricht, wendet dieser diese
Nachricht als Fakt auf eine kontextabh�ngige Regelmenge aus der
Regelbasis an. Dabei werden alle Regeln der Regelmenge
durchlaufen. Gilt eine Bedingung einer Regel, so wird die
zugeh�rige Konsequenz ausgef�hrt. Im Implementierungsteil einer
solchen Konsequenz werden Modifikationen an einem erzeugenden
Workflow-Objekt durchgef�hrt (z.B. wird bei einem unauthorisierten
Commit ein String "'Unauthorisierter Commit"' in das
Workflow-Objekt geschrieben, sofern eine Regel dies definiert
hat).
\par
Ein Workflow-Objekt ist eine XML-Struktur, die einen Titel und eine Liste
aller Regelresultate besitzt. Dieses Workflow-Objekt wird einem Client
zugeteilt und in die Message-Queue des Servers gelegt.
