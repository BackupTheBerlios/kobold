\section{Metadaten}
Kobold verwendet Metadaten um Informationen �ber erstellte
Produktlinien, Produkte und den verwendeten Core-Assets in XML-Dateien
zu persistieren, um diese Informationen an verschiedenen
Arbeitspl�tzen zu verwenden. Diese Dateien werden in einem Repository
gespeichert und stehen unter Versionskontrolle.
%-----------------------------------------------
\subsection{Datenstruktur}
Die Verzeichnisstruktur im Workspace spiegelt den Navigationsbaum
wieder. Dabei werden dynamisch beim deserialisieren der Metadateien
die folgenden Verzeichnisse, die nicht unter Versionskontrolle stehen,
angelegt.

\begin{verbatim}
Workspace/NameDerProduktlinie/
\\				Productline/.productlinemetainfo
\\				Products/NameProdukt/.productmetainfo
\\				CoreAssets/NameCoreAsset/.coreassetmetainfo
\end{verbatim}

Die Datei ``.productlinemetainfo'' beschreibt welche Produkte innerhalb
der Produktlinie liegen. Die aufgef�hrten Produkte werden innerhalb der Datei
``.productmetainfo'' des jeweiligen Produktverzeichnises genau
spezifiziert. Informationen �ber die verwendeten Core-Asset
Komponenten werden durch die Datei ``.coreassetmetainfo'' erfasst.

\subsection{Produktlinien-Metainformation}
Die Datei ``.productlinemetainfo'' enth�lt folgende Informationen �ber
vorhandene Produkte:
\begin{itemize}
  \item{vorhandene Produkte: Name}
  \item{vorhandene Core-Assets: Name}
  \item{Repository-Pfad: Name}
\end{itemize}

\subsection{Produkt-Metainformation}
Die Datei ``.productmetainfo'' enth�lt folgende Informationen �ber
das jeweilige Produkt:
\begin{itemize}
  \item{verwendete Komponente (type=''pspec''): Name}
  \begin{itemize}
    \item{verschiedene Versionen: Name}
    \begin{itemize}
      \item{Filedeskriptor: Pfad, Version der Datei, Letzte
          �nderung an Datei, Letzter Bearbeiter}
    \end{itemize}
    \item{Repository-Pfad: Name}
  \end{itemize}
%--------------------------
  \item{verwendete Komponente (type=''related''): Name}
  \begin{itemize}
    \item{verwendete Variante: Name}

    \begin{itemize}
      \item{verwendete Version: Name}

      \begin{itemize}
	\item{Verweis auf die Version einer Variante der verwendeten
	Core-Asset Komponente}
      \end{itemize}

    \end{itemize}

  \end{itemize}
%--------------------------
\end{itemize}

\subsection{Core-Asset-Metainformation}
Die Datei ``.coreassetmetainfo'' enth�lt folgende Informationen �ber
die verwendeten Core-Assets:
\begin{itemize}
    \item{verwendete Komponente: Name}
    \begin{itemize}
      \item{Varianten: Name}
      \begin{itemize}
        \item{Versionen: Name}
        \begin{itemize}
          \item{Filedeskriptor: Pfad, Version der Datei, Letzte
          �nderung an Datei, Letzter Bearbeiter}
        \end{itemize}
        %------------------
        \item{Komponenten: Name}
        \begin{itemize}
          \item{...}
        \end{itemize}
      \end{itemize}
  \end{itemize}

  \item{Repository-Pfad: Name}

\end{itemize}

