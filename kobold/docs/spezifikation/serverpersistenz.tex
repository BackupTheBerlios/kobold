\section{Serverseitige Persistierung}
Vom Kobold Serverdienst werden Benutzer-, Rollen-, Produkt-,
Produktlinien- und Nachrichtendaten in einem eigenen XML Format
persistiert. Bei Produkten und Produktlinien werden lediglich die Repository-Lokationen
sowie der zugeh�rige Produkt- bzw. Produktlinienname persistiert, da die
wesentliche Metainformations-Speicherung clientseitig geschieht. Alle
anderen Daten werden vollst�ndig serverseitig persistiert.

\subsection{Benutzerpersistierung}

Benutzer werden vom Kobold in Form von folgender XML-Struktur persistiert.

\begin{verbatim}
<users>
    <user>
        <username>vanto</username>
        <realname>Tammo van Lessen</realname>
        <password>87EE3E2F</password>
        <roles>
            <role type="PE">
                <product>xregatta</product>
            </role>
            <role type="P">
                <product>statcvs-xml</product>
            </role>
        </roles>
    </user>
    <user>
        <username>garbeam</username>
        <realname>Anselm R. Garbe</realname>
        <password>FE453E2F</password>
        <roles>
            <role type="PE">
                <product>WMI</product>
            </role>
            <role type="P">
                <product>CAIF</product>
            </role>
        </roles>
    </user>
    ...
</users>
\end{verbatim}

\subsection{Nachrichtendaten}

Nachrichten in Queues oder an spezielle Benutzer werden wie folgt
persistiert:

\begin{verbatim}
<messages>
    <queues>
        <queue>
            <entry>
                <message type="TYP" id="ID" priority="PRIO"
                state="STATE">
                    <sender>vanto</sender>
                    <receiver>garbeam</receiver>
                    <date>200408102�451910</date>
                    <subject>Hallo</subject>
                </message>
            </entry>
            ...
        </queue>
    </queues>
    <pending>
        <message type="TYP" id="ID" priority="PRIO"
        state="STATE">
            <sender>vanto</sender>
            <receiver>garbeam</receiver>
            <date>200408102�451910</date>
            <subject>Hallo</subject>
        </message>
        ...
    </pending>
</messages>
\end{verbatim}

Nachrichten, die global f�r alle verf�gbar sind, werden wie folgt
persistiert:

\begin{verbatim}
<messages>
    <global>
        <message type="TYP" id="ID" priority="PRIO"
        state="STATE">
            <sender>vanto</sender>
            <receiver>garbeam</receiver>
            <date>200408102�451910</date>
            <subject>Hallo</subject>
        </message>
        ...
    </global>
</messages>
\end{verbatim}

\subsection{Produkt- und Produklinien-Daten}

Produkte werden serverseitig unterhalb von Produktlinien
persistiert:

\begin{verbatim}
<productline name="metricline">
    <repository type="CVS">
        <host>cvs.berlios.de</host>
        <path>/cvsroot/metricline</path>
    </repository>
    <products>
        <product name="statcvs-xml">
            <repository type="CVS">
                <host>cvs.berlios.de</host>
                <path>/cvsroot/statcvs-xml</path>
            </repository>
        </product>
        <product name="wmi">
            <repository type="SVN">
                <host>svn.berlios.de</host>
                <path>/svnroot/repos/wmi</path>
            </repository>
        </product>
    </products>
</productline>
\end{verbatim}
