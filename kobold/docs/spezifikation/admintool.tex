\section{Use Cases des Server-Administrations-Tools}

Use Case Name      : Server-Administartions-Tool starten\\
Akteur            : Kobold Administrator\\
Vorbedingung      : -\\
Regul�rer Ablauf  : \\
\begin{itemize}
\item Ein Begrü�&#159;ungstext wird auf der Konsole ausgegeben.
\item Der Kobold Administrator wird aufgefordert, eine gültige URL des
zu administrierenden Servers anzugeben.
\item Der Kobold Administartor gibt eine gültige URL ein.
\item Der Kobold Administrator wird aufgefordert, das gültige
Administrator-Passwort für den zu administrierenden
Kobold Server einzugeben.
\item Der kobold Administrator gibt das gültige Passwort an.
\item Das Administartionstool überprüft die Angaben des Administrators
und gibt eine Erfolgsmeldung aus.
\end{itemize}
Alternative Abl�ufe: \\
Falls keine Verbindung zum angegebenen Server hergestellt werden kann
(Fehleingaben, Kommunikationsproblem),
wird eine entsprechende Fehlermeldung ausgegeben. Der Administartor wird
vor die Wahl gestellt, ob er seine
Eingabe wiederholen möchte, oder ob das Administrationstool beendet
werden soll. \\
Nachbedingung      : Die Eingabeaufforderung der Administartionskonsole
ist zu sehen.\par

Use Case Name      : Server-Administartionstool beenden\\
Akteur            : Kobold Administrator\\
Vorbedingung      : Das Administrationstool ist gestartet und die
Eingabeaufforderung ist zu sehen.\\
Regul�rer Ablauf  : \\
\begin{itemize}
\item Der Koboldadministartor gibt das Kommando "exit" ein. \item
Der Kobold-Server wird beendet.
\end{itemize}
Nachbedingung      : Die Eingabeaufforderung der Administartionskonsole
ist nicht mehr zu sehen, Eingaben 
werden nicht mehr entgegengenommen.\par

Use Case Name      : Kommando�bersicht anzeigen\\
Akteur            : Kobold Administartor\\
Vorbedingung      : Das Administrationstool ist gestartet und die
Eingabeaufforderung ist zu sehen.\\
Regul�rer Ablauf  : \\
\begin{itemize}
\item Der Koboldadministartor gibt das Kommando "help" ein.
\item Eine �bersicht �ber die verf�gbaren Kommandos erscheint auf
der Konsole, mit einer kurzen Beschreibung jedes Kommandos.
\end{itemize}
Nachbedingung      : Die Eingabeaufforderung der Administartionskonsole
ist zu sehen.\par

Use Case Name      : Fehleingabe t�tigen\\
Akteur            : Kobold Administrator\\
Vorbedingung      : Das Administrationstool ist gestartet und die
Eingabeaufforderung ist zu sehen.\\
Regul�rer Ablauf  : \\
\begin{itemize}
\item Der Kobold Administartor gibt ein ung�ltiges Kommando ein.
\item Eine Fehlermeldung wird auf der Konsole ausgegeben.
\end{itemize}
Nachbedingung      : Die Eingabeaufforderung der Administartionskonsole
ist zu sehen.\par

Use Case Name      : Neue Produktlinie anlegen\\
Akteur            : Kobold Administrator\\
Vorbedingung      : Das Administrationstool ist gestartet und die
Eingabeaufforderung ist zu sehen.\\
Regul�rer Ablauf  : \\
\begin{itemize}
\item Der Koboldadministartor gibt das Kommando "createpl" ein. \item
Der Koboldadministartor wird aufgefordert
einen Namen f�r die neue Produktlinie anzugeben. \item Der
Koboldadministrator gibt einen g�ltigen Namen f�r die neue
Produktlinie an. \item Eine Erfolgsmeldung wird ausgegeben.
\end{itemize}
Alternative Abl�ufe: \\
Falls der eingegebene Name ung�ltig ist (leere Eingabe,
Produktlinie bereits vorhanden), wird eine entsprechende
Fehlermeldung ausgegeben und die Bearbeitung abgebrochen.\\
Nachbedingung      : Die Eingabeaufforderung der Administartionskonsole
ist zu sehen.\par

Use Case Name      : Produktlinie entfernen\\
Akteur            : Kobold Administrator\\
Vorbedingung      : Das Administrationstool ist gestartet und die
Eingabeaufforderung ist zu sehen.\\
Regul�rer Ablauf  : \\
\begin{itemize}
\item Der Koboldadministartor gibt das Kommando "removepl" ein. 
\item Der Koboldadministrator wird aufgefordert den Namen der zu
entfernenden Produktlinie einzugeben. 
\item Der Kobold Administrator gibt einen g�ltigen Produktliniennamen
an.
\item Die angegebene Produktlinie wird aus dem System entfernt und eine
kurze Erfolgsmeldung ausgegeben.
\end{itemize}
Alternative Abl�ufe: \\
Falls der eingegebene Produktlinien-Name ung�ltig ist (leere Eingabe,
Produktlinie nicht vorhanden) 
oder die zu entfernende Produktlinie nicht leer ist, wird eine
entsprechende Fehlermeldung ausgegeben 
und die Bearbeitung abgebrochen.\\
Nachbedingung      : Die Eingabeaufforderung der Administartionskonsole
ist zu sehen.\par

Use Case Name      : Neuen PLE anlegen\\
Akteur            : Kobold Administrator\\
Vorbedingung      : Das Administrationstool ist gestartet und die
Eingabeaufforderung ist zu sehen.\\
Regul�rer Ablauf  : \\
\begin{itemize}
\item Der Koboldadministartor gibt das Kommando "addple" ein.
\item Der Koboldadministrator wird aufgefordert, den Namen des Users
einzugeben, dem die neu anzulegende
PLE-Rolle zugewiesen werden soll. 
\item Der Kobold Administrator gibt einen g�ltigen PLE-Namen an. 
\item Der Kobold Administartor wird aufgefordert, die Produktlinie
anzugeben, der der angegebene User als
PLE zugewiesen werden soll.
\item Der Kobold Administrator gibt den Namen der Produktlinie an.
\item Dem angegebenen User und der angegebenen Produktlinie wird die
PLE-Rolle zugewiesen.
\end{itemize}
Alternative Abl�ufe: \\
\begin{itemize}
\item Falls der eingegebene Benutzername ung�ltig ist (z.B. leere
Eingabe), wird eine entsprechende 
Fehlermeldung ausgegeben und die Bearbeitung abgebrochen.
\item Falls der eingegebene Benutzername noch nicht vergeben ist, wird
der Administrator gefragt, ob ein
neuer User mit angegebenen Namen anzulegen, oder die Bearbeitung
abzubrechen ist.
\end{itemize}
Nachbedingung      : Die Eingabeaufforderung der Administartionskonsole
ist zu sehen.\par

Use Case Name      : PLE-Rolle invalidieren\\
Akteur            : Kobold Administrator\\
Vorbedingung      : Das Administrationstool ist gestartet und die
Eingabeaufforderung ist zu sehen.\\
Regul�rer Ablauf  : \\
\begin{itemize}
\item Der Koboldadministartor gibt das Kommando "removeple" ein.
\item Der Koboldadministrator wird aufgefordert den Namen des Users
einzugeben, dessen PLE-Rolle entfernt 
werden soll. 
\item Der Kobold Administrator gibt einen g�ltigen Benutzernamen an. 
\item Der Kobold Administrator wird aufgefordert, den Namen der
Produktlinie anzugeben, für den der angegebene
Benutzer die PLE-Rolle verlieren soll.
\item Der Kobold Administrator gibt einen korrekten Produktliniennamen
ein.
\item Die angegebene PLE-Rolle wird aus dem System entfernt.
\end{itemize}
Alternative Abl�ufe: \\
\begin{itemize}
\item Falls der eingegebene Benutzer-Name ung�ltig ist (leere Eingabe,
Name nicht vergeben), wird eine 
entsprechende Fehlermeldung ausgegeben und die Bearbeitung abgebrochen.
\item Falls der eingegebene Produktlinienname ungültig ist (leere
Eingabe, Produktlinie nicht vorhanden),
wird eine entsprechende Fehlermeldung ausgegeben und die Bearbeitung
abgebrochen.
\end{itemize}
Nachbedingung      : Die Eingabeaufforderung der Administartionskonsole
ist zu sehen.\par
