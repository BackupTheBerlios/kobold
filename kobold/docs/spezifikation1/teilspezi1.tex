\subsubsection{Metainformationskomponente}

Um zus�tzliche Informationen zu den unten aufgef�hrten Elementen in Kobold zu 
speichern, bietet Kobold eine Metainformationskomponente, das hei�t einen Abschnitt 
in Kobold, der die Metainformationen verwaltet. Diese erm�glicht au�erdem eine 
gezielte Suche nach Elementen mit bestimmten Eigenschaften. Metainformationen 
werden auch f�r eine genaue Zuordnung und Identifikation eines Elementes ben�tigt.
Dieses Kapitel liefert eine genaue Auflistung der geplanten Metainformationen.\newline

\paragraph{Produkt:}
Die geplanten Metainformationen f�r diese Iteration sind:
\begin{itemize}
\item Name
\item Liste aller Releases
\end{itemize}

\paragraph{Release:}
Ein Release besitzt folgende Metainformationen:
\begin{itemize}
\item Liste der Objekte
\item Erstellungsdatum
\end{itemize}

\paragraph{Version:}
F�r eine Version werden folgende Metainformationen gespeichert:
\begin{itemize}
\item interne Versionsnummer
\item Status
\end{itemize}

\paragraph{Objekt:}
Ein Objekt ist ein von Menschen geschaffenes Software Objekt wie zum Beispiel 
Quelltext und Dokumentation. Als Metainformationen �ber ein Objekt wird gespeichert:
\begin{itemize}
\item Liste der Versionen
\item Liste der releasef�higen Versionen
\item ID
\item Name
\item bin�r (ja/nein)
\item Beschreibung
\item Zust�ndiger
\end{itemize}

\paragraph{Skript:}
Ein Skript wird bei der angegebenen Aktion ausgef�hrt. Dies kann auch mit 
Parameter�bergabe geschehen. Diese Funktionalit�t wird allerdings erst in 
einer Folgeiteration entwickelt.
\begin{itemize}
\item Aktion
\item Parameter
\item Ausf�hren vor Aktion (ja/nein)
\item Reihenfolge
\end{itemize}

\paragraph{Variante:}
Eine Variante besitzt folgende Metainformationen:
\begin{itemize}
\item Liste aller Objekte
\item Versionsnummer
\item Zust�ndiger
\item Name
\item Beschreibung
\item ID
\item Liste der Skripte
\item Status
\end{itemize}

\paragraph{Komponente:}
F�r jede Komponente werden folgende Metainformationen gespeichert:
\begin{itemize}
\item Liste aller Varianten
\item Name
\item Zust�ndiger
\item Beschreibung
\item ID
\item Liste der Skripte
\item Status
\end{itemize}

\paragraph{Abh�ngigkeit:}
F�r eine Abh�ngigkeit werden folgende Metainformationen gespeichert:
\begin{itemize}
\item Typ
\item Richtung
\item Startknoten
\item Zielknoten
\end{itemize}

\paragraph{Metaknoten:}
F�r Metaknoten werden nur folgende Metainformationen ben�tigt:
\begin{itemize}
\item Typ
\item ID
\end{itemize}

\paragraph{Architektur:}
F�r eine Architektur werden folgenden Metainformationen ben�tigt:
\begin{itemize}
\item Liste der Metaknoten
\item Liste der Abh�ngigkeiten
\item Liste der Komponenten (oberste Ebene)
\item Name
\item Typ
\item Status
\item Zust�ndiger
\item Link auf das Repository
\item Liste der Skripte
\end{itemize}


\subsubsection{VCM Wrapper}

Um die Zugriffsrechte konsistent durchzusetzen, und die automatisierte 
Aktionsbasierte Kommunikation mit dem Kobold Serverdienst zu erm�glichen, 
wird ein VCM Wrapper erstellt. Dieser authentifiziert VCM-Aktionen zwischen dem 
Benutzer und dem VCM des Kobold Serverdienstes und l�st gegebenenfalls einen 
Workflow aus.\par

Der VCM Wrapper behandelt Anfragen vom Kobold Client an den Kobold Server, indem
er diese entweder an das eigentliche VCM weiterleitet oder mit einer passenden 
Meldung ablehnt.\par
Ist im Client noch nicht der Pfad eines Repositories gespeichert, so holt sich 
der VCM Wrapper die n�tigen Informationen vom Kobold Server und speichert diese 
im Client. Er importiert dabei automatisch rollen- und produktabh�ngige 
Repository-Zugriffskonfigurationen.
