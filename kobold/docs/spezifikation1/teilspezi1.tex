\subsubsection{Metainformationen}

Der Begriff Metainformationen umfasst alle Informationen zu
Produkten und Produktlinien, die f�r das \product System notwendig
sind, jedoch weder aus dem zentralen Kobold Serverdienst noch aus
Repository-Informationen gewonnen werden k�nnen.\par

Um die notwendigen Meta-Informationen der unten aufgef�hrten
Elemente in Kobold zu speichern, wird eine
Metainformationskomponente als eigenst�ndiges Plug-In realisiert,
das alle Metainformationen verwaltet. Es erm�glicht au�erdem eine
gezielte Suche nach Elementen mit bestimmten Eigenschaften.
Metainformationen werden auch f�r eine genaue Zuordnung und
Identifikation eines Elementes verwendet.\par Dieses Kapitel
liefert eine genaue Auflistung der geplanten
Metainformationen.\newline

\paragraph{Produkt:}
F�r jedes Produkt werden folgende Metainformationen gespeichert:
\begin{itemize}
\item Name
\item Liste aller Releases
\end{itemize}

\paragraph{Release:}
F�r jedes Release werden folgende Metainformationen gespeichert:
\begin{itemize}
\item Liste der Objekte
\item Erstellungsdatum
\end{itemize}

\paragraph{Version:}
F�r eine Version werden folgende Metainformationen gespeichert:
\begin{itemize}
\item interne Versionsnummer
\item Status
\end{itemize}

\paragraph{Objekt:}
Ein Objekt ist ein beliebiges Softwaredokument (zum Beispiel
Quelltext oder Dokumentation), das in der Regel als Datei
vorliegt. F�r jedes Objekt werden folgende Metainformationen
gespeichert:
\begin{itemize}
\item Liste der Versionen
\item Liste der releasef�higen Versionen
\item ID
\item Name
\item bin�r (ja/nein)
\item Beschreibung
\item Zust�ndiger
\end{itemize}

\paragraph{Skript:}
Die Unterst�tzung f�r Skripte wird erst in einer Folgeiteration
realisiert. Folgende Metainformationen werden daf�r
voraussichtlich gespeichert werden:
\begin{itemize}
\item Aktion
\item Parameter
\item Ausf�hren vor Aktion (ja/nein)
\item Reihenfolge
\end{itemize}

\paragraph{Variante:}
F�r jede Variante werden folgende Metainformationen gespeichert:
\begin{itemize}
\item Liste aller Objekte
\item Versionsnummer
\item Zust�ndiger
\item Name
\item Beschreibung
\item ID
\item Liste der Skripte
\item Status
\end{itemize}

\paragraph{Komponente:}
F�r jede Komponente werden folgende Metainformationen gespeichert:
\begin{itemize}
\item Liste aller Varianten
\item Name
\item Zust�ndiger
\item Beschreibung
\item ID
\item Liste der Skripte
\item Status
\end{itemize}

\paragraph{Abh�ngigkeit:}
F�r jede Abh�ngigkeit werden folgende Metainformationen
gespeichert:
\begin{itemize}
\item Typ
\item Richtung
\item Startknoten
\item Zielknoten
\end{itemize}

\paragraph{Metaknoten:}
F�r jeden Metaknoten werden folgende Metainformationen
gespeichert:
\begin{itemize}
\item Typ
\item ID
\end{itemize}

\paragraph{Architektur:}
F�r jede Architektur werden folgenden Metainformationen
gespeichert:
\begin{itemize}
\item Liste der Metaknoten
\item Liste der Abh�ngigkeiten
\item Liste der Komponenten (oberste Ebene)
\item Name
\item Typ
\item Status
\item Zust�ndiger
\item Link auf das Repository
\item Liste der Skripte
\end{itemize}

\subsubsection{VCM Wrapper}

Um die Zugriffsrechte auf Repositories aus dem Kobold-Client
heraus konsistent durchzusetzen und die automatisierte
Nachrichtenzustellung zum Kobold Serverdienst zu erm�glichen, wird
ein VCM Wrapper erstellt, der das von Eclipse zur Verf�gung
gestellte Repository Abstraction Layer wrappt.\par Der VCM Wrapper
f�ngt VCM-Aktionen an das Repository Abstraction Layer ab und
pr�ft, ob die VCM-Aktionen konform zu den Berechtigungen der
aktuellen Rolle des Benutzers sind. Ist dies der Fall, leitet der
VCM Wrapper die eigentliche VCM-Aktion an das Eclipse Repository
Abstraction Layer weiter. Ist dies nicht der Fall, wird ein
Workflow durch den Server angesto�en und die VCM-Aktion
verweigert.\par Dar�ber hinaus erm�glicht der VCM-Wrapper die
automatische �bernahme von Repository-Zugriffskonfigurationen aus
den Produktlinien- oder Produktkonfigurationen, die vom \product
Serverdienst zur Verf�gung gestellt werden.
