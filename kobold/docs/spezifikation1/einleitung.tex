\chapter{Einleitung}

\section{�ber dieses Dokument}

Diese Spezifikation beschreibt das Rahmensystem und die Basisfunktionalit�ten von 
Kobold, die in der ersten Iteration erstellt werden.\par
Sie ist damit Grundlage f�r die weitere Entwicklung von Kobold innerhalb der 
ersten Iteration. Sie ist auch Grundlage f�r die Absprache mit dem Auftraggeber, 
f�r das Benutzerhandbuch, den Entwurf, die Implementierung und f�r die Tests des Systems.
Die Spezifikation bietet au�erdem eine wichtige Grundlage f�r sp�tere 
Wartungsarbeiten und f�r eventuelle Wiederverwendung.\par
Das Dokument richtet sich an die Mitglieder des Teams sowie an den Auftraggeber 
und dessen technische Berater.\par
Es wird vom Auftraggeber abgenommen und ist von da an Vertragsbestandteil f�r 
die weitere Entwicklung von Kobold und somit auch Grundlage f�r die Abnahme nach 
der ersten Iteration.

\section{Das Kobold-System}

Das wesentliche Einsatzziel des Produktlininen Management Systems Kobold ist die 
werkzeugunterst�tzte Entwicklung und Pflege von Softwareproduktlinien und die 
Etablierung eines rollenbasierten Entwicklungsprozesses.\newline

\subsection{Grundlegende Architekturentscheindungen}
Um Inkonsistenzen im Produkt bzw. der Produktlinie zu vermeiden, ist es n�tig 
eine Zentrale Benutzerverwaltung anzusprechen, die Rechte, Rollen und 
Verpflichtungen der Benutzer bereitstellt. Einem Benutzer k�nnen mehrere Rollen 
zugeordnet werden. Kobold unterst�tzt drei unterschiedliche Rollen: Den 
Produktlinieningenieur, den Produktingenieur und den Programmierer. \par
Dem Produktlinieningenieur untersteht eine Produktlinie. 
Er verwaltet au�erdem die Rechte seiner Produktingenieure. Diese verwalten ein 
einzelnes Produkt einer Produktlinie. Die dritte und letzte Rolle ist die des  
Programmierers, der dem Produktingenieur untersteht und dessen Produkte 
implementiert.\newline

\subsection{Client-Server Architektur}
Die Architektur von Kobold besteht aus zwei Teilen: Dem Kobold Client und dem 
Kobold Serverdienst. \newline
Der Client bietet dem Benutzer eine graphische Benutzeroberfl�che, �ber die er 
seine Produktlinien und seine Produkte verwalten kann. Er kann damit seine 
Architekturen ansehen und ver�ndern, neue Rollen verteilen, 
Nachrichten verschicken, etc. (Genaueres wird in Kapitel 2 betrachtet). 
Die Architektur wird in einer Graphenform dargestellt.\newline

Der Server verwaltet die Daten der einzelnen Benutzer und deren Zugriffsrechte. 
Er bietet den Clients au�erdem einen eigenen Nachrichtendienst. F�hrt der Client 
eine bestimmte Aktion durch, meldet er diese dem Server, der sie dann auf 
m�gliche Inkonsistenzen, die dabei auftreten k�nnen, untersucht. Der Server 
verwaltet auch die Pfade der Repositories, in denen die Daten der Produkte und 
Produktlinien gespeichert und versioniert werden.