\subsubsection{Drools}

Kobold bietet seinem Benutzer einen Workflow-Mechanismus, der die Wahrung des 
definierten produktlinien-basierten Entwicklungsprozesses unterst�tzt. Dieser 
f�hrt Konsistenzpr�fungen durch, die in anderen Client-Instanzen automatisch 
Workflows  ansto�en, um die Inkonsistenz zu beheben. Zur Realisierung des 
Workflow-Mechanismus wird die Rule Engine Drools\footnote{http://www.drools.org} 
verwendet.\par
Daf�r wird eine Regelbasis in XML f�r Drools erzeugt. Diese besteht aus einer 
Menge aller Regelmengen und befindet sich auf dem Server. Eine Regelmenge 
wiederum besteht aus Regeln. Eine solche Regel gliedert sich in drei Teile: 
Parameter, Bedingung und Konsequenz. Dabei bilden die Parameter eventuell ben�tigte 
Parameter f�r die Regel. Die Bedingung �berpr�ft, ob die Regel angewendet wird 
oder nicht. Die Konsequenz beeinhaltet die Aktionen, die durchgef�hrt werden, 
falls die Bedingung erf�llt ist.\par

Der Workflow-Mechanismus spielt sich sowohl auf dem Kobold Server als auch auf 
dem Kobold Client ab.\par

\paragraph{Aktionen auf dem Kobold Server}
F�hrt ein Client eine Aktion aus, so wird diese dem Kobold Server gemeldet. 
Dieser wendet die Aktion als Fakt auf die Regelbasis an, die sich auf dem Server 
befindet. Ein Fakt ist dabei eine Mitteilung an den Server. Es werden dabei 
alle Regeln durchlaufen. Stimmt das Ereignis mit der Bedingung einer Regel 
�berein, so wird die Konsequenz dieser ausgef�hrt. Die Ergebnisse dieser 
Regelkonsequenzen werden zu einem Workflow-Objekt zusammengef�gt (zum Beispiel 
eine Reihe von Anweisungen). Ein Workflow-Objekt ist dabei eine Art XML-Struktur, 
die einen Titel und eine Liste aller Regelresultate besitzt. Diese Workflow-Objekt 
wird dann auf die Message-Queues der Clients gelegt, f�r die das Workflow-Objekt 
relevant ist. 

\paragraph{Aktionen auf dem Kobold Client}
Sobald der Client eine Aktion ausf�hrt, meldet er diese dem Kobold Server um 
m�gliche Inkonsistenzen zu vermeiden. Wird ein Workflow entwickelt, so ruft 
jeder Kobold Client, der den Workflow ansto�en sollte, beim n�chsten Login 
dieses Objekt ab. Der Anwender wird dar�ber in der Workflow-View des Kobold 
Clients informiert.\newline
\newline
\includegraphics[width=15cm]{drools.png}\newline

\paragraph{Die Regelbasis in Kobold}
Kobold wird in dieser Iteration eine Anfangsregelbasis zur Verf�gung stellen, 
die vom Anwender jederzeit erweitert werden kann. Er muss dabei eine neue 
Regel erstellen und sie der Regelbasis hinzuf�gen. Diese Regeln m�ssen dann 
nicht unbedingt zum Workflow-Objekt hinzugef�gt werden, sondern k�nnen in 
ihrer Konsequenz auch andere Aktionen ausf�hren. 
Dies bleibt dem Anwender �berlassen. Zur Erstellung einer neuen Regel wird 
ihm au�erdem ein Editor zur Verf�gung stehen, der in einer der kommenden 
Iterationen entstehen wird. 