\chapter{Nichtfunktionale Anforderungen}
\section{Leistungsanforderungen}
Das System wird zun�chst so ausgelegt, dass mind. 50 Benutzer gleichzeitig mit 
dem Produkt arbeiten k�nnen. Die Skalierbarkeit h�ngt ausschliesslich vom Server 
ab, dessen Leistungsf�higkeit von der Hardware und der verwendeten 
Persistenzschicht- Implementierung beeinflusst wird. Garantiert wird die 
Bereitstellung von 10 Produktlinien zu je 50 Produkten. Pro Produktlinie k�nnen 
100 Benutzer verwaltet werden.

\section{Minimale Hardwareanforderungen}
Das Softwaresystem wird auf einem Pentium II basierten PC mit 400 Mhz CPU-Takt, 
256 MB Hauptspeicher und mind. 200 MB freien Festplattenspeicher (oder unter 
einer vergleichbaren Unix-Workstation) unter den Betriebssystemen Windows, Solaris 
und Linux lauff�hig sein.

\section{Entwurfseinschr�nkungen}
Das Produkt wird in Java 2 Version 1.4 implementiert. Als Workbench-Framework wird die 
Eclipse Plattform\footnote{http://www.eclipse.org/, http://www.eclipse.org/platform/}
verwendet, sowie deren Widgettoolkit SWT\footnote{Standard Widget Toolkit, 
http://www.eclipse.org/swt/} 
und GEF\footnote{Graphical Editing Framework, http://www.eclipse.org/gef/}.
Zudem werden Bibliotheken von der Apache Software
Foundation\footnote{http://www.apache.org/} benutzt. Die
Generierung der Dokumenation setzt auf iText\footnote{http://www.lowagie.com/iText/} auf.
F�r die SSL-verschl�sselte Kommunikation mit dem Server kommt die JSSE von Sun zum Einsatz.
\footnote{Java Secure Socket Extension, http://java.sun.com/products/jsse/index.jsp}

\section{Verf�gbarkeit}
F�r den Client gibt es keine besonderen Anforderungen bzgl. der Verf�gbarkeit. 
Der Kobold-Serverdienst soll mit mind. 99\% hoch verf�gbar sein. 
Die Server-Verf�gbarkeit h�ngt trotzdem von der Konfiguration des Rechners, 
der den Serverdienst zur Verf�gung stellt, ab.

\section{Sicherheit}
Die vom Server bereitgestellten Daten werden bei jeder �nderung sofort auf dem 
Datentr�ger gespeichert. Somit wird die Gefahr eines Datenverlusts beim Eintritt 
von unvorhersehbaren Ereignissen minimiert.
Alle client-spezifischen Daten werden in den Produktlinien- bzw.
Produkt-Repositories gespeichert und unterliegen  den Sicherheitsregeln des verwendeten
Versionskontrollsystems.

\section{Robustheit}
Fehleingaben und Fehler im System werden erkannt und dem Benutzer mit einer aussagekr�ftigen 
Fehlermeldung mitgeteilt. Sie f�hren nicht zum Programmabsturz. Genaueres folgt 
in einer sp�teren Iteration.

\section{Wartbarkeit}
Durch die st�ndige �berpr�fung des Programmquellcodes mit Metriken auf Kopplung, 
Zusammenhalt und Styleguide-Konformit�t wird die hohe Wartbarkeit des 
Produkts gew�hrleistet.

\section{Performance}
%Das Softwaresystem soll laut Kundenangaben auf einem handels�blichen
%Pentium II PC mit mind. 400 Mhz CPU-Takt, mind. 256 MB Hauptspeicher
%und 200 MB freien Festplattenspeicher (oder auf einer vergleichbare Unix-Workstation)
%unter den Betriebssystemen Windows, Solaris oder Linux lauff�hig sein.
%Das System soll es erm�glichen die Produkt(-linien) Architekturen als Graph zu
%visualisieren.
%Es wurde nicht gefordert den ganzen Graphen, der unter Umst�nden sehr komplex und 
%gro� werden kann, auf einmal performant zu visualisieren.\par
%Es gen�gt die f�r die 
%jeweilige Architekturansicht relevanten Teile des Graphen zu visualisieren.
%Diese Ausschnitte werden sich der �bersichtlichkeit halber wahrscheinlich in einer 
%Gr��enordnung bis zu 125 Knoten bewegen. Die Anzeige der Teilgraphen sollte 
%eine Grenze von 8 Sekunden zur Berechnung der Darstellung nicht �berschreiten.
Es wurde nicht gefordert den ganzen Graphen, der unter Umst�nden sehr komplex
werden kann, performant zu visualisieren.\par
Es soll gen�gen, die f�r die jeweilige Architekturansicht relevanten
Ausschnitte des Graphen zu visualisieren.
Diese werden sich der �bersichtlichkeit pro forma in einer 
Gr��enordnung bis zu 125 Knoten bewegen. Die Anzeige der Teilgraphen sollte 
eine Grenze von 8 Sekunden zur Berechnung der Darstellung nicht �berschreiten.

\section{Portabilit�t}
Der Server st�tzt sich auf keine nativen Schnittstellen sondern benutzt ausschlie�lich 
\emph{pure java}. Somit ist er laut Sun Microsystems 
Inc.\footnote{unter http://java.sun.com/j2se/1.4.2/system-configurations.html} unter 
Solaris-SPARC, Solaris-Intel, Windows NT/2000/XP und Linux lauff�hig.

Der Client basiert grundlegend auf der Eclipse Plattform und deren Widgettoolkit und ist
dadurch von dessen nativer Schnittstelle abh�ngig. Laut dem Eclipse Consortium werden die 
folgenden Plattformen und Betriebssysteme unterst�tzt:
\begin{itemize}
    \item Windows NT/2000/XP
    \item Linux (Motif)
    \item Linux (GTK 2)
    \item Solaris 8 (SPARC/Motif)
    \item QNX (x86/Photon)
    \item AIX (PPC/Motif) 
    \item HP-UX (HP9000/Motif)
    \item Mac OSX (Mac/Cocoa)
\end{itemize}
