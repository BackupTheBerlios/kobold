\section{GXL Import und Export Funktionalit�t}
Da bei der Erstellung dieses Dokumentes noch kein �bereinkommen mit 
dem Auftraggeber getroffen werden konnte, und dieses auch erst in den n�chsten Wochen m�glich sein wird,
wird in diesem Fall nur eine sehr grobe Spezifikation vorgenommen. 
Grunds�tzlich ist vorgesehen, dass der GXL Import / Export in einem separaten Plug-In realisiert werden wird, das zum Feature-Set geh�rt. 
Es ist vorgesehen, sowohl die Daten des Architekturgraphen, als auch die 
Daten, die auf Dateiebene hinter der Darstellung stehen, exportieren zu k�nnen. 
So werden bei einem Export die Architektur-Daten in Form einer XML Datei, 
die dem mit dem Kunden noch zu spezifizierenden GXL Schema entspricht, gespeichert.
Die mit der Architektur zusammenh�ngenden Assets werden in einem .jar File exportiert. Dies schlie�t
Meta Informationen aus.