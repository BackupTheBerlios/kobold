\chapter{User Interface}

\section{Architecture Editor}

The Architecture Editor displays the architecture of the product,
or product line respectively, that is selected in the Role View. \par

Architectures are directed graphs that consist of nodes and edges. 
Nodes represent components, variants and files. Directed edges represent
any relationship between nodes.\par

Through drag and drop you can move the elements of the architecture.

\section{Role View}

The Role View shows the current projects that are checked out in your workspace 
(see \ref{roletree}).
A project can either be a product or a product line. You can navigate between the 
different projects by selecting them. The open tree in the Role View displays the
role of the current user in the project. You can also see the architecture and its 
elements.

\begin{figure}[h!]
\begin{center}
\includegraphics[width=10cm]{roletree.png}
   \caption{Role View}
\label{roletree}
\end{center}
\end{figure}\par

When you right-click on a project, a context menu will open where you can choose 
between different actions (see \ref{rolekontext}).

\begin{figure}[h!]
\begin{center}
\includegraphics[width=10cm]{rolekontext.png}
   \caption{Project Context Menu}
\label{rolekontext}
\end{center}
\end{figure}\par

\section{Workflow/Task View}

In the bottom of the window you can see the Workflow/Task View which displays all 
the messages (K) and workflows (W) for the current user (see \ref{workflow}). 

\begin{figure}[h!]
\begin{center}
\includegraphics[width=15cm]{workflow.png}
   \caption{Workflow/Task View}
\label{workflow}
\end{center}
\end{figure}\par

Double-click on an entry and a 
separate dialog will be opened where you can read the details of that message or
workflow (see \ref{workflowdialog}).

\begin{figure}[h!]
\begin{center}
\includegraphics[width=5cm]{workflowdialog.png}
   \caption{Message Dialog}
\label{workflowdialog}
\end{center}
\end{figure}\par

When you right-click on a message, a context menu will open where you can choose 
between different actions (see \ref{workflowkontext}).

\begin{figure}[h!]
\begin{center}
\includegraphics[width=10cm]{workflowkontext.png}
   \caption{Message Context Menu}
\label{workflowkontext}
\end{center}
\end{figure}\par

\section{Minimap}

The Minimap is on the left side beneath the Role View. It shows the whole architecture.
By clicking at one spot on the map, the Architecture Editor automatically centers on that
spot. 
