\chapter{Installation}

\section{Technical Requirements}
Kobold runs on 

\begin{itemize}
\item Windows NT/2000/XP
\item Linux (Motif)
\item Linux (GTK 2)
\item Solaris 8 (SPARC/Motif)
\item Mac OSX (Mac/Cocoa)
\end{itemize}

with the following requirements: \par

\begin{itemize}
\item 600 MHz
\item 256 MB main memory
\item 200 MB fixed-disk storage
\end{itemize}
\par


Its server is purely based on Java and therefore runs on:
\begin{itemize}
\item Solaris-SPARC
\item Windows NT/2000/XP
\item Linux
\end{itemize}


% Oli: bitte hier dann einf�gen --> bitteschoen:-)
\section{Installation}
To make distributing and installation easier a set of files and the
necessary make shell script is located at
/home/stsopra/werkbold/install\_scripts (change ME) for an easy
linux installation. It also includes the binaries of java, eclipse and
perl (also for windows).
%
\subsection{Create a distribution iso}
To create a distribution iso the make.sh script is executed with the
argument ``distribute''. The script creates:
\begin{itemize}
\item the distributing directory
\item the iso-file directory
\item the generated iso-file
\end{itemize}
The generated iso-file is used to burn a cd-rom or can be easily
mounted into the VM-ware.
%
\subsection{Install Kobold}
To install Kobold, the generated iso-file is needed. Mount the
iso-file and change into the install-directory. The make.sh script is
executed with the argument ``install'' and the used install path of
the user.
All needed files are installed to the install path and the start
scripts for the server, the sat-tool and the kobold application are
created and adapted to the choosen install path.
\subsection{Adapt the PATH-variable}
After the installation the user can add the shown path to his
PATH-variable to start the tools without using the full path.
The tcsh path is extended in the file ~/.cshrc with the command ``setenv PATH KOBOLD\_HOME:PATH''.
\subsection{Start the tools}
The tools are started with the start-scripts located in the bin directory:
\begin{itemize}
\item start-server.sh to start the server
\item start-sat.sh to start the server administartion tool
\item start-kobold.sh to get Kobold running
\end{itemize}
The start script initialize java and perl of the Kobold package and
set other settings.
