\chapter{Instructions for Installation}

After you have started Eclipse 2.1, change to CVS Repositories perspective and add a CVS repository. \par
Enter the following data (see \ref{neuesrep}):
\begin{itemize}
\item host: cvs.berlios.de
\item repository path: /cvsroot/kobold
\item user: anonymous
\end{itemize}
Leave the remaining data untouched and confirm the dialog. \par

\begin{figure}[h!]
\begin{center}
\includegraphics[width=12cm]{neuesrep.png}
   \caption{Adding a new repository}�
\label{neuesrep}
\end{center}
\end{figure}

Open the tree along HEAD, kobold and src (see \ref{auschecken}). Check out the four folders kobold.client.plam,
kobold.client.vcm, kobold.common and kobold.server through 'check out as..' (context menue)
and confirm with 'Finish'.

\begin{figure}[h!]
\begin{center}
\includegraphics[width=10cm]{auschecken.png}
   \caption{This is how your tree should look like}
\label{auschecken}
\end{center}
\end{figure}\par

Switch to Java perspective and open up the kobold.client.plam tree. Double-click plugin.xml.
A form opens with 'required plugins' on the right. Press 'more' and delete 'org.eclipse.ui.ide'.
Switch to the 'source' tab and change all entries from ui.ide to ui. Save plugin.xml. \par
Open the 'Window' menu and select 'preferences'. Select 'plug-in development' and 'Target 
Platform' and press the button 'Not in Workspace'. Confirm with OK. \par

Projects kobold.client.plam, kobold.client.vcm and kobold.common: \par
Right-click on the project and select 'properties'. Select 'Java build path' and the 'source'
tab. Remove the existing entry and add a new one by pressing 'Add Folder'. Choose the 
src-folder and confirm. Append '/bin' to the default output folder (see \ref{buildpath}).
 Close the preferences dialog. \par

\begin{figure}[h!]
\begin{center}
\includegraphics[width=15cm]{buildpath.png}
   \caption{This is how your buildpath should look like}
\label{buildpath}
\end{center}
\end{figure}\par

Projects kobold.server: \par
Right-click on the project and select 'properties'. Select 'Java build path' and the 'source'
tab. Remove the existing entry and add a new one by pressing 'Add Folder'. Choose the 
src-folder and confirm. Append '/bin' to the default output folder. 
Switch to the 'projects' tab and select 'kobold.common'. Close the preferences 
dialog. \par

Again, right-click on the project kobold.client.plam and select 'update classpath'. Select 
kobold.client.plam, kobold.client.vcm and kobold.common and confirm.
