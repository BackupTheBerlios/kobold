\chapter{Overview of Kobold}

The essential architecture of Kobold is divided in two subsystems: the Kobold Client
which is a development environment for productlines and products, and the Kobold 
Server which administrates the users, roles and messages. \par

The Client is based on Eclipse and offers the user a graphical user interface
through which he can administrate his productlines and products. He can view and alter the
architectures in form of graphs, send messages and trigger workflows.\par

The Server administrates the data of each user and their responsibilities. Also it
offers the Clients a central message service . If a message is sent from a Client
to the Server, the Server checks the message for possible consequences which are
assigned to other clients as workflows. Besides the Server administrates the paths
and the access configurations of the repositories which save the data of the products
and the productlines.

\section{Architecture Objects}

\subsection{Core Asset}

Core Assets are abstract modules in the productline architecture. They can contain variants.

\section{Variant}

Whenever there are more than one design possibilites, a variant is created. It allows
the user to pursue both possibilites.

\section{Release}

Releases are created to save the state of a variant at a specific date. 

\subsection{Related Component}
Related components are 

\subsection{Specific Component}

Specific components are instances of releases and used in product architectures. They can
differ from the related components in order to fit the product.

\subsection{Dependency Edge}

A dependency edge indicates that the nodes connected by the edge depend on each other.
Therefore one of them can't be used without the other.

\subsection{Exclusion Edge}

An exclusion edge indicates that the nodes connected by the edge can't be used
at the same time.

\subsection{Meta Node}

Meta nodes are used to create complex relationships between nodes. They represent
the relationships AND and OR. In an AND relationship all objects are needed. The OR
relationship lets you decide between the objects of the relationship.
Be careful with
the direction of the edges that are linked to a meta node! In the following architecture component A
needs components B and C. But components B and C are independent of component A (see \ref{example}).

\begin{figure}[h!]
\begin{center}
\includegraphics[width=10cm]{example.png}
   \caption{Meta node example}
\label{example}
\end{center}
\end{figure}\par


\subsection{GXL}

GXL is a graph exchange language and used to export the architectures in Kobold. For more
information: http://www.gupro.de/GXL/


\section{Roles}

\subsection{Productline Engineer (PLE)}

Productline engineers have the responsibility for the productlines.

\subsection{Product Engineer (PE)}

Product engineers administrate products and core assets and are subordinates of the
productline engineers.

\subsection{Programmer (P)}

Programmers are subordinates of the product engineers. They develop the software of the
products and core assets.

\section{Productline}

Kobold is a tool to administrate productlines. By specifying an architecture in which
core assets are linked with each other through dependency and exclusion edges, the productline engineer
creates a basis for all the products of the productline. 

\section{Workflow}

A workflow is a working process. You can create them in order to trigger certain
activities when the server executes one of the following commands:

\begin{itemize}
	\item add user
	\item get all users
	\item update user password
	\item update user full name
	\item get productline names
	\item get productline
	\item update productline
\end{itemize}

In Kobold workflows are triggered by either WorkflowMessage or RPCSpy objects. You can
find the interfaces and implementations in chapter 6.