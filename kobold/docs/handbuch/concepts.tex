\chapter{The concepts in Kobold}

\section{Productline}

Kobold is a tool to administrate productlines. By specifying an architecture in which
core assets are linked with each other through dependency and exclusion edges, the product line engineer
creates a basis for all the products of the product line. 

\section{Component}

Components are abstract modules in the productline architecture. Instances of them (i.d. variants and custom components)
are used for the different products of the productline.

\section{Custom Component}

Custom components are instances of components and used in product architectures. They can
differ from the corresponding components in order to fit the product.

\section{Variant}

Variants are modifications of components.

\section{Release}

Releases are created to save the current state of a variant. 

\section{Meta Node}

Meta nodes are a means to creating complex relationships between nodes. They represent
the relationships AND and OR.

\section{Dependency Edge}

A dependency edge indicates that the nodes connected by the edge depend on each other.
Therefore one of them can't be used without the other.

\section{Exclusion Edge}

An exclusion edge indicates that the nodes connected by the edge can't be used
in the same product.

\section{GXL}

GXL is a graph exchange language and used to export the architecture, etc. For more
information: http://www.gupro.de/GXL/

\section{Product Line Engineer (PLE)}

The product line engineer has the responsibility for the product line.

\section{Product Engineer (PE)}

The product engineer administrates a specific product and is a subordinate of the
product line engineer.

\section{Product Developer (P)}

Product developers are subordinates of the product engineer. They develop the software 
product.

\section{Workflow}

A workflow is a working process. You can create them in order to trigger certain
activities when the server executes one of the following commands:

\begin{itemize}
	\item login
	\item logout
	\item get roles
	\item get product role
	\item add user
	\item get productline
	\item get product
	\item add product
	\item add role
	\item remove role
	\item apply productline modifications
	\item apply product modifications
	\item remove user
\end{itemize}