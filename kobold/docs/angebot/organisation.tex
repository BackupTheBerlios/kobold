%%%%%%%%%%%%%%%%%%%%%%%%%%%%%%%%%%%%%%%%%%%%%%%%%%%%%%%%%%%%%%%%%%%%%%%%%%%%%%%
%% StuPro A, Produktlinien (Kobold)
%% Team Werkbold
%% Angebot
%% $Id: organisation.tex,v 1.1 2004/01/28 18:10:49 garbeam Exp $
%%%%%%%%%%%%%%%%%%%%%%%%%%%%%%%%%%%%%%%%%%%%%%%%%%%%%%%%%%%%%%%%%%%%%%%%%%%%%%%

\section{Organisation des Teams}

\subsection{Team}

Das Team f�r die \product Entwicklung mu� verschiedene Rollen wahrnehmen, die
im Folgenden erl�utert werden. Angestrebt wird aufgrund des evolution�ren
Vorgehensmodells und der Resultate der ersten Iteration die Organisation in
Form eines {\it Chief-Programmer Teams}\footnote{Vgl. Literatur, u.a. J.
Ludewig, Software-Engineering f�r Softwaretechniker, Vorlesungsfolien},
in dem ein oder zwei Team-Mitglieder die Rolle als Chief-Programmer wahrnehmen,
die den roten Faden in der Entwicklung vorgeben.

\subsection{Rollen}

\begin{itemize}
\item {\bf Projektleiter:} Der Projektleiter vertritt die Interessen des Teams
nach au�en gegen�ber dem Auftraggeber, koordiniert die Projektplanung,
�berwacht den Entwicklungsfortschritt und f�rdert den Informationsaustausch
im Team und zum Auftraggeber.
\item {\bf Chief-Programmer:} Der Chief-Programmer fungiert als technischer
Projektleiter, der die technischen Aspekte der Implementierungsphasen vorgibt,
das Team technisch ber�t und die technische Realisierung �berwacht. Er
entwickelt ma�geblich die Basisfunktionalit�ten.
\item {\bf Configuration-Manager:} Der Configuration-Manager setzt die
Entwicklungsumgebung auf und stellt die Verwaltung und Versionierung von Dokumenten
und des Source-Codes sicher.
\item {\bf QS-Manager:} Der QS-Manager erstellt Richtlinien f�r Dokumente und
den Source-Code, die f�r alle Team-Mitglieder verbindlich sind. Weiterhin
erstellt er Checklisten, die helfen beim Abschluss der einzelnen Phasen und am
Ende einer Iteration die Qualit�t der Resultate zu pr�fen. Dar�ber hinaus
ist der QS-Manager f�r den Testplan verantwortlich und koordiniert die
Testdurchf�hrung.
\item {\bf Programmierer:} Der Programmierer implementiert und dokumentiert in
R�cksprache mit dem Chief-Programmer die ihm zugeteilten Komponenten bzw.
Arbeitspakete. Dar�ber hinaus schreibt er Testf�lle f�r seine Implementierungen.
\item {\bf Dokumentations-Manager:} Der Dokumentations-Manager ist technischer
Berater f�r alle Dokumente, die in der Entwicklung enstehen. Er ist f�r die
Erstellung des Handbuchs verantwortlich.
\end{itemize}


\section{Ansprechpartner}

Auftraggeber:
\begin {itemize}
  \item Daniel Simon\\
    Telefon: +49-711-7816-213\\
    E-Mail: simon@informatik.uni-stuttgart.de\\
\end {itemize}

Qualit�tssicherung und Controlling:
\begin {itemize}
  \item Thomas Eisenbarth\\
    Telefon: +49-711-7816-345\\
    E-Mail: eisenbarth@informatik.uni-stuttgart.de\\
\end {itemize}

Technischer Berater:
\begin {itemize}
  \item J�rg Czeranski\\
    Telefon: +49-711-7816-317\\
    E-Mail: czeranski@informatik.uni-stuttgart.de\\
  \item Gunther Vogel\\
    Telefon: +49-711-7816-375\\
    E-Mail: gunther.vogel@informatik.uni-stuttgart.de\\
\end {itemize}

Werkbold-Team:
\begin {itemize}
  \item Anselm Garbe (Projektleiter)\\
    Telefon: +49-711-8822280\\
    E-Mail: anselmg@t-online.de
  \item Necati Aydin\\
    Telefon: +49-177-6029410\\
    E-Mail: neco.a@gmx.de
  \item Oliver Rendgen\\
    Telefon: +49-711-717619\\
    E-Mail: stuproa@hypeo.de
  \item Patrick Schneider\\
    Telefon: +49-163-2788161\\
    E-Mail: PatrickM.Schneider@gmx.de
  \item Tammo van Lessen\\
    Telefon: +49-711-6586471\\
    E-Mail: tvanlessen@taval.de
\end {itemize}

%%% Local Variables: 
%%% TeX-master: "angebot"
%%% End: 
%%% vim:tw=79:
