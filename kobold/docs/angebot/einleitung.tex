%%%%%%%%%%%%%%%%%%%%%%%%%%%%%%%%%%%%%%%%%%%%%%%%%%%%%%%%%%%%%%%%%%%%%%%%%%%%%%%
%% StuPro A, Produktlinien (Kobold)
%% Team Werkbold
%% Angebot
%% $Id: einleitung.tex,v 1.1 2004/01/28 18:10:49 garbeam Exp $
%%%%%%%%%%%%%%%%%%%%%%%%%%%%%%%%%%%%%%%%%%%%%%%%%%%%%%%%%%%%%%%%%%%%%%%%%%%%%%%

%@Tammo: �nderungen?
%
\chapter {Einleitung}

Dieses Angebot zur Projektdurchf�hrung im Rahmen des Studienprojekt A
``Produktlinien'' der Firma \company richtet sich stellvertretend an Herrn Dipl.-Inf. 
Daniel Simon, der im Folgenden als {\it Auftraggeber} der Firma {\it Bauhaus
Reengineering} von Prof. Dr. Pl�dereder (Abteilung Programmiersprachen und
�bersetzerbau an der Universit�t Stuttgart) bezeichnet wird.\par
Gegenstand dieses Angebotes ist die Entwicklung des Produktlinien
Management Systems \product.

\section{Ziel}
Dieses Dokument beschreibt alle Vorraussetzungen und Ziele f�r die Entwicklung des
Produktlinien Management Systems \product, um alle gew�nschten Anforderungen des
Auftraggebers zu realisieren. Es wird dem Auftraggeber als Angebot f�r
das Hauptprojekt vorgelegt.\par
Das wesentliche Einsatzziel des Produktlininen Management Systems \product
ist die werkzeugunterst�tzte Entwicklung und Pflege von
Softwareproduktlinien und die Etablierung eines rollenbasierten
Entwicklungsprozesses. Dieser soll eine effiziente
Produktlinienentwicklung erm�glichen und wird durch \product unterst�tzt.

\section{G�ltigkeit}

Der in diesem Angebot beschriebene Leistungskatalog gilt bis einschlie�lich
{\bf 31. Januar 2004}. Das Angebot muss bis zu diesem Zeitpunkt von allen
beteiligten Personen unterzeichnet worden sein, andernfalls verliert es seine
G�ltigkeit.

\section{Entwicklungsphilosophie}

Da auf dem Markt kein vergleichbares Produktlinien-Management System
existiert, ist \product eine vollst�ndige Erstentwicklung.\par
Erstentwicklungen ist wesenseigen, dass sich einige Anforderungen
im Laufe des Entwicklungsprozesses �ndern oder neu ergeben k�nnen.
So ist es
z.B. m�glich, dass sich durch die Evaluation der implementierten
Anforderungen neue sinnvolle Erg�nzungen ergeben, die man nicht
vorhersagen konnte.\par
Deshalb wird ein evolution�res Vorgehensmodell angewendet, das sich in mehrere
Iterationen unterteilt. Jede Iteration folgt dem Wasserfall-Vorgehensmodell.
Die erste Iteration gew�hrleistet die Erstellung eines soliden
Rahmensystems und stellt die Basisfunktionalit�t gem�� den sp�ter
aufgef�hrten Anforderungen zur Verf�gung.\par
In den weiteren Iterationen werden die kundenspezifischen Anforderungen nach
einer mit dem Auftraggeber abzustimmenden Priorisierung implementiert.

\section{Weiteres Vorgehen}

Im Rahmen dieses Angebotes wird die Firma \company in der 2. Kalendwerwoche
2004 eine Angebotspr�sentation durchf�hren. Inhalt dieser Pr�sentation wird die
Vorstellung des Prototypen sowie die grundlegende Erl�uterung der
Anforderungen und der Architekturentscheidungen sein.\par
Zur Angebotspr�sentation wird dem Kunden ein erg�nzendes Dokument �ber den
\product Prototypen zur Verf�gung gestellt.

%%% Local Variables: 
%%% TeX-master: "angebot"
%%% End: 
%%% vim:tw=79:
