%%%%%%%%%%%%%%%%%%%%%%%%%%%%%%%%%%%%%%%%%%%%%%%%%%%%%%%%%%%%%%%%%%%%%%%%%%%%%%%
%% StuPro A, Produktlinien (Kobold)
%% Team Werkbold
%% Angebot
%% $Id: lieferumfang.tex,v 1.1 2004/01/28 18:10:49 garbeam Exp $
%%%%%%%%%%%%%%%%%%%%%%%%%%%%%%%%%%%%%%%%%%%%%%%%%%%%%%%%%%%%%%%%%%%%%%%%%%%%%%%

\section {Lieferumfang}

Resultate des Vorprojektes werden im Rahmen dieses Angebotes und der
Angebotspr�sentation (in der 2. Kalenderwoche 2004) dem Auftraggeber
ausgeliefert.\par
Die Resultate des Hauptprojektes werden im Rahmen der Endauslieferung des
fertigen \product Systems am {\bf 13. August 2004} dem Auftraggeber
�bergeben.\par

Folgende Dokumente und Software-Pakete werden im Vor- und Hauptprojekt
ausgeliefert:

\subsection {Vorprojekt}

\begin{itemize}
\item Angebot (vorliegendes Dokument)
\item Dokument zur Angebotspr�sentation
\end{itemize}

\subsection {Hauptprojekt}

\subsubsection{Iteration I}

Ziel der ersten Iteration ist die Auslieferung eines lauff�higen Rahmensystems,
das die gesamte Basisfunktionalit�t bereits enth�lt. Im Einzelnen werden folgende Dokumente
und Software-Pakete ausgeliefert:

\begin{itemize}
\item Spezifikation der Basisfunktionalit�t
\item Entwurf der Basisfunktionalit�ten (Client-Server Architektur)
\item Source- und Bytecode des Clients und Servers
\item Source- und Bytecode notwendiger Komponenten von Drittanbietern
\item Testf�lle, Testplan und Testresultate des Clients und Servers
\item Installations-Howto
\end{itemize}

\subsubsection{Folge-Iterationen}

Ziel der Folge-Iterationen ist die Auslieferung einer lauff�higen Version, in
der priorisierte Anforderungen implementiert wurden und die somit
das \product System St�ck f�r St�ck vervollst�ndigen. Dadurch kann der
Entwicklungsfortschritt durch den Auftraggeber besonders gut beurteilt werden.
Im Einzelnen werden i.d.R. folgende Dokumente und Software-Pakete ausgeliefert:

\begin{itemize}
\item Spezifikation der neu zu implementierenden Anforderungen
\item Entwurf der neu zu implementierenden Anforderungen, sofern nicht durch
Basisfunktionalit�ten abgedeckt
\item Source- und Bytecode der integrierten neuen Anforderungen (Client und
Server)
\item Source- und Bytecode notwendiger Komponenten von Drittanbietern
\item Testf�lle, Testplan und Testresultate der neuen Anforderungen
\item Weitergehende Howtos, sofern erforderlich
\end{itemize}

\subsubsection{Letzte Iteration}

Ziel der letzten Iteration ist die Endauslieferung von \product.
Diese umfasst folgende Dokumente und Software-Pakete:

\begin{itemize}
\item Source- und Bytecode des Clients und Servers
\item Source- und Bytecode aller notwendigen Komponenten von Drittherstellern
\item Source- und Bytecode des Server-Administrationstools
\item Zusammenstellung aller Spezifikationsdokumenten aus den einzelnen
Iterationen
\item Zusammenstellung aller Entwurfsdokumente aus den einzelnen Iterationen
\item Zusammenstellung aller Entwickler-Howtos
\item Zusammenstellung aller Testf�lle, Testpl�ne und Testresultate
\item Handbuch
\item Installationsanleitung
\item Dokumente der Abschlusspr�sentation
\item Projektabschlussbericht
\end{itemize}

\subsubsection{Nachbesserungsphase}

In dieser Phase k�nnen Nachbesserungen zur Zufriedenheit des Kunden
vorgenommen werden.

\subsection{Andere Resultate}
Alle anderen Dokumente, die w�hrend des Projekts entstehen, geh�ren nicht zum Lieferumfang.

%%% Local Variables: 
%%% TeX-master: "angebot"
%%% End: 
%%% vim:tw=79:
