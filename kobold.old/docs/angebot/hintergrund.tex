%%%%%%%%%%%%%%%%%%%%%%%%%%%%%%%%%%%%%%%%%%%%%%%%%%%%%%%%%%%%%%%%%%%%%%%%%%%%%%%
%% StuPro A, Produktlinien (Kobold)
%% Team Werkbold
%% Angebot
%% $Id: hintergrund.tex,v 1.1 2004/01/28 18:10:49 garbeam Exp $
%%%%%%%%%%%%%%%%%%%%%%%%%%%%%%%%%%%%%%%%%%%%%%%%%%%%%%%%%%%%%%%%%%%%%%%%%%%%%%%

\chapter{Produkt}
\section{Hintergrund der Produktlinienentwicklung}

Schl"usselerfolgsfaktoren in der Softwaretechnik sind kurze
Time-to-market Zyklen, hohe Produktqualit"at und niedrige Kosten. Das Erzielen
dieser scheinbar unvereinbaren Ziele wird durch systematische
Wiederverwendung w"ahrend der Entwicklung von Software m"oglich.\par

% Die
%Erfahrung hat gezeigt, dass systematische Wiederverwendung durch die
%Einf"uhrung von produktlinienorientierter Software-Entwicklung gut
%unterst"utzt werden kann.\par

Kernpunkt des Produktlinienansatzes ist die systematische Erfassung
von Unterschieden und Gemeinsamkeiten der verschiedenen Produkte und
einer expliziten Abbildung dieser Aspekte in die
Softwarearchitektur. Diese Grundidee hat Auswirkungen f"ur alle
Schritte im Produktentwicklungsprozess und - idealerweise - auch
dar"uber hinaus in der Produktplanung und -strategie.\par

Um eine effiziente Produktlinienentwicklung in der Softwareentwicklung
zu etablieren, ist es wichtig, dass diese Produktstrategie konsequent
umgesetzt wird.\par

Idealziel einer gut konfigurierten Produktlinie ist es, ein Produkt
durch einfache Kombination von angepassten Core-Assets zu
erstellen. Eine Produktlinie ist also ideal konfiguriert, wenn die
Produktlinien-Architektur f�r alle m�glichen Produkte hinreichend
spezifiziert ist. Der Produkt\-linien-Ingenieur hat genau
diese Aufgabe. Er bestimmt die Architektur in dem er Core-Assets
definiert und deren Grundbeziehungen zueinander spezifiziert (Scoping
und Domain-Engineering). Soll nun ein neues Produkt in die
Produktlinie aufgenommen werden (Application Engineering), so muss es
dieser Architektur folgen. Es werden dazu bereits existierende
Varianten der Core-Assets mit neuen Varianten kombiniert und dem
Produkt-Ingenieur �bergeben, der die Verantwortung f�r das Produkt
�bernimmmt. Dieser hat daf�r Sorge zu tragen, dass die Entwicklung an
dem Produkt nicht die Architektur der Produktlinie verletzt.\par 

In dieser Hierarchie ist es wichtig, dass bestimmte Vorg�nge nach
festen Regeln kommuniziert werden. Wenn z.B. in einem Produkt eine
bereits existierende Variante eines Core-Assets verwendet wird, darf
diese nicht von dem Produkt-Entwickler ver�ndert werden. Ver�nderungen
sind hier nur dem zust�ndigen Core-Asset-Entwickler gestattet. Dieser
muss abw�gen, ob die Ver�nderung sinnvoll ist, da er die
Verantwortung f�r sein Modul, welches evtl. auch noch von anderen
Produkten verwendet wird, hat. Lehnt er dies ab, so muss der
Produkt-Ingenieur eine produktspezifische Variante dieses Core-Assets
entwickeln lassen. Dieser Vorgang ist sehr komplex, da mehrere Personen
mit ihren Entscheidungen daran beteiligt sind. Um eine Produktlinie
konsequent durchsetzen zu k�nnen, m�ssen diese Arbeitsabl�ufe
spezifiziert sein.\par

Es bietet sich nun an, die Entwicklung von Produktlinien mit Werkzeugen zu
unterst�tzen. Damit kann man sowohl das Configuration Managment
vereinfachen, sowie die oben erw�hnten Arbeitsabl�ufe modellieren und damit
die Kommunikation und die Entwicklung erleichtern.

