\chapter{Server}
Der Server wird als HTTP-basierter {\it XML-RPC Server}
\footnote{N�here Informationen: http://ws.apache.org/xmlrpc/}
implementiert und ist SSL-basiert. Ausserdem erm�glicht
der Server eine Client-zu-Client-Kommunikation �ber eine Nachrichten-Queue, welche
in regelm�ssigen Abst�nden von den Clients abgefragt wird (Request/Response).
Alle Informationen, die der Server bereitstellt sind persistent.\par
Die Wartung des Servers wird durch ein kommandozeilen-orientiertes
Administrationstool realisiert.\newline

F�r jeden Benutzer werden folgende Daten gespeichert:
\begin{itemize}
\item Benutzername
\item Passwort
\item eMail
\item Name
\item Liste von Rollen
\end{itemize}

Eine Rolle besteht aus einer Liste von Produkten und Produktlinien.\newline
F�r jedes Element in der Liste wird au�erdem noch eine Liste der zugeh�rigen Repositories angeh�ngt. 
Ist das Element ein Produkt, so werden auch die Daten der dazugeh�rigen Produktlinie erfasst.\newline

F�r ein Repository werden folgende Daten gespeichert:
\begin{itemize}
\item Pfad
\item Passwort
\item Benutzername
\item Schreibrecht (ja/nein)
\end{itemize}

\section{Berechtigungskonzept}

Zur Realisierung des produktlinien\-�bergreifenden Berechtigungskonzepts
verwaltet der Server alle benutzer-, rollen-, produktlinien- und
produkt-basierten Berechtigungen. Dabei kann ein Server produktlinien-�bergreifend
verwendet werden.\par
Der Server bietet ein auf diesen Berechtigungen basierenden
Authentifizierungsmechanismus an, der von allen Clients verwendet wird.

\section{Persistierung}

Die Persistierung aller f�r das Berechtigungskonzept notwendigen Daten wird
durch eine abstrakte Persistenzschicht realisiert, die es erm�glicht, die
Datenhaltung flexibel zu organisieren. Im Rahmen dieses Angebotes wird die
Persistierung XML-basiert angeboten.

\section{Nachrichten-Queue}

Die vom Server angebotene Nachrichten-Queue ist zustandsbehaftet, d.h. nach einem
Stromausfall oder anderen unvorhersehbaren Ereignissen ist die Nachrichten-Queue
in der Regel ohne Datenverlust wiederherstellbar.\par
Dadurch wird ein hoher Grad an Konsistenztreue und Zuverl�ssigkeit erreicht.\newline

F�r die Nachrichten-Queue werden folgende Daten ben�tigt:
\begin{itemize}
\item von wem
\item an wen
\item Inhalt
\item Datum
\item ID
\item Priorit�t
\end{itemize}



\section{Web-basierte Statusinformationen}

Es ist jederzeit f�r einen authentifizierten Systemadministrator m�glich,
den Status des Servers �ber einen Webbrowser abzufragen.

\section{Administrationstool}

Zur Administration des Servers wird ein kommandozeilen-orientiertes 
Administrationstool angeboten, dass es erm�glicht, Produktlinien-Ingenieur-Accounts
anzulegen bzw. zu entfernen, die Nachrichten-Queue zu
leeren und den Server zu starten oder zu stoppen.