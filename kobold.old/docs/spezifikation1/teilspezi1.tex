\section{Metainformationskomponente}

F�r die Elemente in Kobold werden unterschiedliche Metainformationen gespeichert. Dieses Kapitel liefert eine genaue Auflistung der geplanten Metadaten.\newline

\paragraph{Produkt:}
Das Produkt wird vom Produktingenieur bearbeitet und entsteht aus der Architektur der Produktlinie. 
\begin{itemize}
\item letztes Release
\end{itemize}

\paragraph{Release:}
Ein Release besteht aus einer Gruppe von Objekten (Source Code, Dokumentation, etc.) in einem Zustand, in dem sie als Teil des fertigen Produktes eingesetzt werden kann.
\begin{itemize}
\item Liste der Dateien (mit Versionsangabe)
\item Liste der zus�tzlichen Objekte
\item Liste der Skripte
\item Erstellungsdatum
\end{itemize}

\paragraph{Version:}
Objekte (Source Code, Dokumentation, etc.) sind in unterschiedlichen Versionen verf�gbar. Die Versionskontrolle �bernimmt dabei eine Standard Versionsverwaltungssystem wie zum Beispiel CVS, RCS, etc.
\begin{itemize}
\item Liste der Skripte
\item Status
\end{itemize}

\paragraph{Datei:}
Eine Datei ist ein Objekt, das Source Code enth�lt.
\begin{itemize}
\item Liste der Versionen
\item ID
\item Name
\item bin�r (ja/nein)
\item Beschreibung
\end{itemize}

\paragraph{Variante:}
Eine Variante besteht entweder aus weiteren Komponenten oder aus Releases.
\begin{itemize}
\item Liste aller Dateien
\item Versionsnummer
\item Zust�ndiger
\item Name
\item Beschreibung
\item ID
\item Liste der Skripte
\item Status
\end{itemize}

\paragraph{Komponente:}
Eine Komponente besteht aus einer oder mehreren Varianten.
\begin{itemize}
\item Liste aller Varianten
\item Name
\item Zust�ndiger
\item Beschreibung
\item ID
\item Liste der Skripte
\item Status
\end{itemize}

\paragraph{Abh�ngigkeit:}
Eine Abh�ngigkeit besteht zwischen zwei Knoten der Architektur. F�r Abh�ngigkeiten zwischen mehr als zwei Knoten werden Metaknoten verwendet.
\begin{itemize}
\item Typ
\item Richtung
\item Knoten1
\item Knoten2
\item weiter beliebig attributierbar
\end{itemize}

\paragraph{Metaknoten:}
Metaknoten werden f�r Mehrfachbeziehungen verwendet.
\begin{itemize}
\item Typ
\item ID
\end{itemize}

\paragraph{Architektur:}
Eine Architektur stellt sowohl ein Produkt als auch eine Produktlinie graphisch dar.
\begin{itemize}
\item Liste der Metaknoten
\item Liste der Abh�ngigkeiten
\item Liste der Komponenten (oberste Ebene)
\item Name
\item Typ
\item Status
\item Zust�ndiger
\item Link auf das Repository
\item Liste der Skripte
\end{itemize}


\section{Repository Abstraction Layer:}

Zur besseren Kommunikation mit der Server, soll ein Repository Abstraction Layer erstellt werden, das mit der Authentifizierung am Server umgehen kann. Dieses vermittelt zwischen Kobold und dem VCM der Eclipse-Komponente.\newline

Das Repository Abstraction Layer behandelt Anfragen an den Server, indem es diese entweder an das eigentliche VCM weiterleitet, oder mit einer passenden Meldung ablehnt.\par
Ist in der Eclipse-Komponente noch nicht der Pfad eines Repositories gespeichert, so holt sich das Repository Abstraction Layer die n�tigen Informationen vom Server und speichert diese in der Eclipse-Komponente.
