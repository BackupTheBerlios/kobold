\subsubsection{Metainformationen}

Als Metainformationen werden alle Informationen zu Produkten und
Produktlinien verstanden, die f�r das \product System notwendig sind,
jedoch weder auf dem zentrlen Kobold Serverdienst noch aus
Repository-Informationen gewonnen werden k�nnen.\par

Um Meta-Informationen zu den unten aufgef�hrten Elementen in Kobold zu
speichern, bietet Kobold eine Metainformationskomponente als
eigenst�ndiges Plug-In, das die Metainformationen verwaltet. Dieses
erm�glicht au�erdem eine gezielte Suche nach Elementen mit bestimmten
Eigenschaften. Metainformationen werden auch f�r eine genaue Zuordnung
und Identifikation eines Elementes ben�tigt.\par
Dieses Kapitel liefert eine genaue Auflistung der geplanten Metainformationen.\newline

\paragraph{Produkt:}
Ein Produkt besitzt folgende Metainformationen:
\begin{itemize}
\item Name
\item Liste aller Releases
\end{itemize}

\paragraph{Release:}
Ein Release besitzt folgende Metainformationen:
\begin{itemize}
\item Liste der Objekte
\item Erstellungsdatum
\end{itemize}

\paragraph{Version:}
F�r eine Version werden folgende Metainformationen gespeichert:
\begin{itemize}
\item interne Versionsnummer
\item Status
\end{itemize}

\paragraph{Objekt:}
Ein Objekt ist ein von Menschen geschaffenes Software Objekt wie zum Beispiel Quelltext und Dokumentation. 
I.d.R. ist ein Objekt eine Datei.
Als Metainformationen �ber ein Objekt wird gespeichert:
\begin{itemize}
\item Liste der Versionen
\item Liste der releasef�higen Versionen
\item ID
\item Name
\item bin�r (ja/nein)
\item Beschreibung
\item Zust�ndiger
\end{itemize}

\paragraph{Skript:}
Ein Skript wird bei der angegebenen Aktion ausgef�hrt. Dies kann auch mit Parameter�bergabe geschehen. 
Diese Funktionalit�t wird allerdings erst in einer Folgeiteration entwickelt.
\begin{itemize}
\item Aktion
\item Parameter
\item Ausf�hren vor Aktion (ja/nein)
\item Reihenfolge
\end{itemize}

\paragraph{Variante:}
Eine Variante besitzt folgende Metainformationen:
\begin{itemize}
\item Liste aller Objekte
\item Versionsnummer
\item Zust�ndiger
\item Name
\item Beschreibung
\item ID
\item Liste der Skripte
\item Status
\end{itemize}

\paragraph{Komponente:}
F�r jede Komponente werden folgende Metainformationen gespeichert:
\begin{itemize}
\item Liste aller Varianten
\item Name
\item Zust�ndiger
\item Beschreibung
\item ID
\item Liste der Skripte
\item Status
\end{itemize}

\paragraph{Abh�ngigkeit:}
F�r eine Abh�ngigkeit werden folgende Metainformationen gespeichert:
\begin{itemize}
\item Typ
\item Richtung
\item Startknoten
\item Zielknoten
\end{itemize}

\paragraph{Metaknoten:}
F�r Metaknoten werden nur folgende Metainformationen ben�tigt:
\begin{itemize}
\item Typ
\item ID
\end{itemize}

\paragraph{Architektur:}
F�r eine Architektur werden folgenden Metainformationen ben�tigt:
\begin{itemize}
\item Liste der Metaknoten
\item Liste der Abh�ngigkeiten
\item Liste der Komponenten (oberste Ebene)
\item Name
\item Typ
\item Status
\item Zust�ndiger
\item Link auf das Repository
\item Liste der Skripte
\end{itemize}

\subsubsection{VCM Wrapper}

Um die Zugriffsrechte auf Repositories aus dem Kobold-Client heraus 
konsistent durchzusetzen und die automatisierte aktionsbasierte
Nachrichtenzustellung zum Kobold Serverdienst zu erm�glichen, wird ein
VCM Wrapper erstellt, welche die von Eclipse zur Verf�gung gestellte
Repository Abstraction Layer wrappt.\par
Der VCM Wrapper f�ngt VCM-Aktionen an die Eclipse Repository Abstraction
Layer ab und pr�ft, ob die VCM-Aktion konform zu den Berechtigungen der
ausf�hrenden Rolle sind. Ist dies der Fall, leitet der VCM Wrapper
die eigentliche VCM-Aktion an das Eclipse Repository Abstraction Layer
weiter. Ist dies nicht der Fall, wird ein Workflow durch den Server
angesto�en und die VCM-Aktion verweigert.\par
Dar�ber hinaus bietet der VCM-Wrapper die automatische �bernahme von
Repository-Zugriffskonfigurationen, die aus einer Produktlinien- oder
Produktkonfiguration vom Kobold Serverdienst zur Verf�gung gestellt
werden, in die Eclipse-Repository Konfiguration.
