\subsection{Metainformationskomponente}

Um zus�tzliche Informationen zu den unten aufgef�hrten Elementen in Kobold zu speichern, bietet Kobold eine 
Metainformationskomponente, das hei�t einen Abschnitt in Kobold, der die Metainformationen verwaltet. Diese 
erm�glicht au�erdem eine gezielte Suche nach Elementen mit bestimmten Eigenschaften. Metainformationen 
werden au�erdem f�r eine genaue Zuordnung und Identifikation eines Elementes ben�tigt.
Dieses Kapitel liefert eine genaue Auflistung der geplanten Metainformationen.\newline

\paragraph{Produkt:}
Das Produkt wird vom Produktingenieur bearbeitet. Dieser l�sst sich vom Produktlinieningenieur die Varianten 
und Abh�ngigkeiten zuordnen, die er f�r sein Produkt ben�tigt. Diese Funktionalit�t wird allerdings erst 
in einer Folgeiteration entwickelt. Die geplanten Metainformationen f�r diese Iteration sind:
\begin{itemize}
\item Name
\item Liste aller Releases
\end{itemize}

\paragraph{Release:}
Ein Release besteht aus einer Gruppe von Objekten (Source Code, Dokumentation, etc.). Diese werden in einem 
Release zusammengefasst, wenn sie zusammen ein Produkt bilden, das den Anforderungen f�r eine 
Ver�ffentlichung gen�gt. Ein Release wird manuell vom Produktlinieningenieur oder vom Produktingenieur 
gesetzt. Diese Funktionalit�t wird allerdings erst in einer Folgeiteration entwickelt. Ein Release besitzt 
folgende Metainformationen:
\begin{itemize}
\item Liste der Objekte (mit Versionsangabe)
\item Erstellungsdatum
\end{itemize}

\paragraph{Version:}
Objekte sind in unterschiedlichen Versionen verf�gbar. Die Versionskontrolle �bernimmt dabei eine Standard 
Versionsverwaltungssystem wie zum Beispiel CVS, RCS, etc. F�r eine Version werden folgende Metainformationen 
gespeichert:
\begin{itemize}
\item interne Versionsnummer
\item Status
\end{itemize}

\paragraph{Objekt:}
Ein Objekt ist ein von Menschen geschaffenes Software Objekt wie zum Beispiel Quelltext uns Dokumentation. 
Als Metainformationen �ber ein Objekt wird gespeichert:
\begin{itemize}
\item Liste der Versionen
\item Liste der releasef�higen Versionen
\item ID
\item Name
\item bin�r (ja/nein)
\item Beschreibung
\item Zust�ndiger
\end{itemize}

\paragraph{Skript:}
Ein Skript wird bei der angegeben Aktion ausgef�hrt. Dies kann auch mit Parameter�bergabe geschehen. 
Diese Funktionalit�t wird allerdings erst in einer Folgeiteration entwickelt.
\begin{itemize}
\item Aktion
\item Parameter
\item Ausf�hren vor Aktion (ja/nein)
\end{itemize}

\paragraph{Variante:}
Eine Variante besteht entweder aus weiteren Komponenten oder aus Releases. Sie besitzt dabei immer 
folgende Metainformationen:
\begin{itemize}
\item Liste aller Dateien
\item Versionsnummer
\item Zust�ndiger
\item Name
\item Beschreibung
\item ID
\item Liste der Skripte
\item Status
\end{itemize}

\paragraph{Komponente:}
Eine Komponente besteht aus einer oder mehreren Varianten. Zus�tzlich werden f�r jede Komponente noch 
folgende Metainformationen gespeichert:
\begin{itemize}
\item Liste aller Varianten
\item Name
\item Zust�ndiger
\item Beschreibung
\item ID
\item Liste der Skripte
\item Status
\end{itemize}

\paragraph{Abh�ngigkeit:}
Eine Abh�ngigkeit besteht zwischen zwei Knoten der Architektur. F�r Abh�ngigkeiten zwischen mehr als zwei 
Knoten werden Metaknoten verwendet. F�r eine Abh�ngigkeit werden folgene Metainformationen gespeichert:
\begin{itemize}
\item Typ
\item Richtung
\item Knoten1
\item Knoten2
\item weiter beliebig attributierbar
\end{itemize}

\paragraph{Metaknoten:}
Metaknoten werden f�r Mehrfachbeziehungen verwendet. F�r sie werden nur folgende Metainformationen ben�tigt:
\begin{itemize}
\item Typ
\item ID
\end{itemize}

\paragraph{Architektur:}
Eine Architektur stellt sowohl ein Produkt als auch eine Produktlinie graphisch dar. Zum verlustfreien 
Speichern eine Architektur werden folgenden Metainformationen ben�tigt:
\begin{itemize}
\item Liste der Metaknoten
\item Liste der Abh�ngigkeiten
\item Liste der Komponenten (oberste Ebene)
\item Name
\item Typ
\item Status
\item Zust�ndiger
\item Link auf das Repository
\item Liste der Skripte
\end{itemize}


\subsection{Repository Abstraction Layer:}

Zur besseren Kommunikation mit dem Kobold Server wird ein Repository Abstraction Layer erstellt. 
Dieses authentifiziert VCM-Aktionen zwischen Kobold und dem VCM des Kobold Clients.\newline

Das Repository Abstraction Layer behandelt Anfragen vom Kobold Client an den Kobold Server, indem es diese 
entweder an das eigentliche VCM weiterleitet oder mit einer passenden Meldung ablehnt.\par
Ist im Client noch nicht der Pfad eines Repositories gespeichert, so holt sich das 
Repository Abstraction Layer die n�tigen Informationen vom Kobold Server und speichert diese im Client. 
Es importiert dabei automatisch rollen- und produktabh�ngige 
Repository-Zugriffskonfigurationen.
