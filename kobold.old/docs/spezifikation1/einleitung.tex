\chapter{Einleitung}

\section{�ber dieses Dokument}

Diese Spezifikation beschreibt das Rahmensystem und die Basisfunktionalit�ten von Kobold, die im Laufe der ersten Iteration erstellt werden sollen.\par
Sie ist damit Grundlage f�r die weitere Entwicklung von Kobold innerhalb der ersten Iteration. Darunter fallen auch die Absprache mit dem Auftraggeber, das Benutzerhandbuch, der Entwurf, die Implementierung und der Test des Systems.
Die Spezifikation bietet au�erdem eine wichtige Grundlage f�r sp�tere Wartungsarbeiten und f�r eventuelle Wiederverwendung.\par
Das Dokument ist an die Mitglieder des Teams sowie an den Auftraggeber und dessen technische Berater gerichtet.\par
Nach Fertigstellung des Dokuments wird dieses vom Auftraggeber abgenommen. Es ist von da an Vertragsbestandteil f�r die weitere Entwicklung von Kobold und somit auch Grundlage f�r die Abnahme nach der ersten Iteration.

\section{�ber Kobold}

Das wesentliche Einsatzziel des Produktlininen Management Systems Kobold ist die werkzeugunterst�tzte 
Entwicklung und Pflege von Softwareproduktlinien und die Etablierung eines rollenbasierten 
Entwicklungsprozesses. Dieser soll eine effiziente Produktlinienentwicklung erm�glichen und wird durch 
Kobold unterst�tzt.\newline

Kobold unterst�tzt drei unterschiedliche Rollen: Dem Produktlinieningenieur untersteht eine Produktlinie. 
Er verwaltet au�erdem die Rechte seiner Produktingenieure. Diese verwalten ein einzelnes Produkt einer Produktlinie. Die dritte und letzte Rolle ist der 
Programmierer, der dem Produktingenieur untersteht und dessen Produkte implementiert.\newline

Die Architektur von Kobold gliedert sich in zwei Teile: Client und Server. \newline
Der Client bietet dem Benutzer eine graphische Benutzeroberfl�che, �ber die er seine Produktlinien und seine 
Produkte verwalten kann. Er kann damit seine Architekturen ansehen und ver�ndern, neue Rollen verteilen, 
Nachrichten verschicken, etc. (N�heres siehe Kapitel 2).\newline

Der Server verwaltet die Daten der einzelnen Benutzer und deren Zugriffsrechte. Er bietet den 
Clients au�erdem einen eigenen Nachrichtendienst. F�hrt der Client eine bestimmte Aktion durch, meldet er diese dem Server, 
der sie dann auf m�gliche Inkonsistenzen, die dabei auftreten k�nnen, untersucht. Der Server verwaltet auch die Pfade der Repositories, in 
denen die Daten der Produkte und Produktlinien gespeichert und versioniert werden. 