\chapter{Einleitung}

\section{�ber dieses Dokument}

Diese Spezifikation dient als Grundlage zur informellen
Beschreibung des Produktlinien Management Systems \product.
Durch das evolution�re Vorgehensmodell wird dieses Dokument
iterativ erweitert, verfeinert und zu Beginn jeder
Folge-Iteration der weiteren Entwicklung zugrundegelegt.\par Das
Dokument richtet sich sowohl an den Auftraggeber und dessen
technische Berater als auch an die Mitarbeiter des
Werkbold-Teams.\par Es wird vom Auftraggeber am Anfang jeder
Iteration abgenommen und ist von da an Vertragsbestandteil f�r die
weitere Entwicklung von \product.

\subsection{Spezifikation I}
Die Spezifikation I dient als Grundlage zur informellen
Beschreibung des Rahmensystems von \product, das in der ersten
Iteration entwickelt wird.\par Sie basiert auf der Abstraktion der
ermittelten Anforderungen aus dem internen Anforderungsdokument
f�r \product \footnote{Siehe
ftp://ftp.berlios.de/pub/kobold/docs/anforderungen/anforderungen.pdf},
das st�ndig angepasst wird.\par
Aufgrund des abstrakten Charakters
der Spezifikation des Rahmensystems wird auf eine explizite
Spezifikation von Use-Cases verzichtet. Diese wird Gegenstand der
Spezifikationen in den Folge-Iterationen sein.

\section{Das Kobold-System}

Das wesentliche Einsatzziel des Produktlinien Management Systems
\product ist die werkzeugunterst�tzte Entwicklung und Pflege von
Software-Produktlinien und die Etablierung eines rollenbasierten
Entwicklungsprozesses.


\subsection{Grundlegende Architekturentscheidungen}
Die grundlegende Architektur von \product untergliedert sich in
zwei Teilsysteme: der Kobold Client, der als rollenbasierte
Entwicklungsumgebung f�r die Verwaltung von Produktlinien und
Produkten konzipiert ist, und der Kobold Serverdienst, der die
Benutzer-, Rollen- und Nachrichten-Verwaltung erm�glicht. \par Der
Eclipse-basierte Client bietet dem Benutzer eine graphische
Benutzungsoberfl�che, �ber die er seine Produktlinien und Produkte
rollenabh�ngig verwalten kann. Er kann damit seine Architekturen
als Graphen ansehen und ver�ndern, neue Rollen verteilen,
Nachrichten verschicken und Workflows ausl�sen.\par Der Server
verwaltet die Daten der einzelnen Benutzer und deren
Zugriffsrechte. Er bietet den Clients au�erdem einen zentralen
Nachrichtendienst an. Wenn von einem Client eine Nachricht an den
Serverdienst gesendet wird, pr�ft der Serverdienst die Nachricht
auf m�gliche Konsequenzen, die anderen Clients in Form von
Workflows zugeteilt werden. Dar�ber hinaus verwaltet der
Serverdienst auch die Pfade und Zugriffskonfigurationen der
Repositories, in denen die Daten der Produkte und Produktlinien
gespeichert und versioniert werden.
