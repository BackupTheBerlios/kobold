\chapter{Der Kobold Client}
Der Client basiert grundlegend auf der Eclipse Plattform und deren Widgettoolkit und ist
dadurch von dessen nativer Schnittstelle abh�ngig. Laut dem Eclipse Consortium werden die folgenden 
Plattformen und Betriebssysteme unterst�tzt:
\begin{itemize}
    \item Windows NT/2000/XP
    \item Linux (x86/Motif)
    \item Linux (x86/GTK 2)
    \item Solaris 8 (SPARC/Motif)
    \item QNX (x86/Photon)
    \item AIX (PPC/Motif) 
    \item HP-UX (HP9000/Motif)
    \item Mac OSX (Mac/Carbon)
\end{itemize}
Er wird als Feature-Set implementiert und mit eigenem Product Branding versehen.
Das Product Branding umfasst die �nderung der Fensternamen und der Produkticons, sowie
einer Willkommensseite die einen kurzen �berblik �ber Funktionalit�t und Zweck des Kobold Tools geben soll.

Das Feature-Set wird als Set von inernationalisierbaren Eclipse Plugins implementiert.
Die Ausgangsperspektive wird aus 4 Teilen bestehen:
\begin{itemize}
	\item Der Produktlinienarchitektur View/Editor
	\item Der Rollen View
	\item Der Worklflow/Task View
	\item Die Minimap
\end{itemize}
Diese werden in den jeweiligen Unterkapiteln n�her beschrieben.
Um das Rollen
\section{Der Produktlinienarchitektur View}
In diesem View wird die 
\section{Der Rollen View}
\section{Der Worklflow/Task View}
\section{Die Minimap}